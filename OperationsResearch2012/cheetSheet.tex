%%% Local Variables: 
%%% mode: latex
%%% TeX-master: t
%%% End: 

\documentclass[12pt,a4paper]{article}
\usepackage[ukrainian]{babel}
\usepackage[utf8]{inputenc}
\usepackage[T2A]{fontenc}
\usepackage[left=2cm,top=2cm,right=2cm,bottom=2cm,nohead,nofoot]{geometry}
\usepackage{setspace}

\setcounter{tocdepth}{1}

\newenvironment{slim_enumerate}{
\begin{enumerate}
  \setlength{\itemsep}{1pt}
  \setlength{\parskip}{0pt}
  \setlength{\parsep}{0pt}}
{\end{enumerate}}

\newenvironment{slim_itemize}{
\begin{itemize}
  \setlength{\itemsep}{1pt}
  \setlength{\parskip}{0pt}
  \setlength{\parsep}{0pt}}
{\end{itemize}}

\newenvironment{slim_description}{
\begin{description}
  \setlength{\itemsep}{1pt}
  \setlength{\parskip}{0pt}
  \setlength{\parsep}{0pt}}
{\end{description}}

\begin{document}

\pretolerance=-1
\tolerance=6500

\pagestyle{empty}

\tableofcontents
\clearpage

\setstretch{1}
\fontsize{14pt}{6mm}\selectfont

\section{Задача про мінімальний каркас. Алгоритм Пріма.}

Нехай дано простий зв’язний навантажений граф $G=(V,E)$ і вагова функція $d:E\rightarrow R$.

Потрібно знайти мінімальний каркас $A_s$ в заданому графі, починаючи з вершини $x_s$.

Алгоритм Пріма:
\begin{slim_enumerate}
  \item Нехай $T_s = \{x_s\}$ - множина вершин, з’єднаних ребрами, що входять в мінімальний каркас,\\
$A_s = \{\emptyset\}$ - множина ребер, що входять в каркас мінімальної довжини.
  \item Записати:\\
$\forall x_j\in$ Г$(x_s)$ $[\alpha_j=x_s, \beta_j=d(x_s,x_j)]$ (Г$(x_s)$ - суміжні до $x_s$ вершини)\\
$\forall x_j\notin$ Г$(x_s)$ $[0,\infty]$
  \item Вибрати $x_j^*$, де $\beta_j^*=\displaystyle\min_{x_j\notin T_s}\{\beta_j\}$,\\
$T_s=T_s\cup\{x_j^*\}$,\\
$A_s=A_s\cup\{(\alpha_j^*,x_j^*)\}$.\\
Якщо $|T_s|=n\Rightarrow$ кінець,\\
інакше ${\Rightarrow}$ Крок 4.
  \item $\forall x_j\notin T_s, x_j\in$ Г$(x_j^*), \beta_j>d(x_j^*,x_j)$ оновити мітки:\\
$\beta_j=d(x_j^*,x_j), \alpha_j=x_j^*$.\\
Перейти на Крок 3.
\end{slim_enumerate}

*місце на граф (малюнок-приклад)*

\clearpage

\section{Формулювання задач про найкоротший шлях. Знаходження найкоротшого шляху від заданої вершини (алгоритм Форда).}

\clearpage

\section{Знаходження найкоротших шляхів між будь-якими вершинами графа (алгоритм Флойда).}

\clearpage

\section{Задача про призначення.}

\clearpage

\section{Задача про максимальний потік. Теорема Форда-Фалкерсона.}

\clearpage

\section{Планування на мережах. Основні поняття та визначення.}

\clearpage

\section{Планування на мережах. Структура та правила побудови.}

\clearpage

\section{Формулювання транспортної задачі. Властивості транспортної задачі.}

\clearpage

\section{Опорні плани транспортної задачі та їх властивості.}

\clearpage

\section{Побудова початкових опорних планів транспортної задачі. Метод мінімального елемента.}

\clearpage

\section{Відкрита та закрита моделі транспортної задачі.}

\clearpage

\section{Критерій лінійної незалежності системи векторів $A_{ij}$ умов транспортної задачі.}

\clearpage

\section{Побудова початкових опорних планів транспортної задачі. Метод Фогеля.}

\clearpage

\section{Критерій розкладу довільного вектора $A_{kl}$ через систему лінійно незалежних векторів $A_{ij}$ умов транспортної задачі.}

\clearpage

\section{Двоїста задача. Умови оптимальності.}

\clearpage

\section{Метод потенціалів розв'язування транспортної задачі.}

\clearpage

\section{Метод диференціальних рент.}

\clearpage

\section{Транспортна задача за критерієм часу.}

\clearpage

\section{Методи гілок і меж. Загальна схема.}

\clearpage

\section{Задача комівояжера.}

\clearpage

\section{Методи гілок і меж для задачі комівояжера.}

\clearpage

\section{Методи гілок і меж для задач лінійного цілочислового програмування.}

\clearpage

\section{Основні поняття теорії ігор. Класифікація ігор.}

Гра є математичною моделлю реальної конфліктної ситуації. \emph{Теорія Ігор} - розділ математики, в якому досліджуються питання поведінки і виробляються оптимальні правила, стратегії поведінки для кожного із учасників гри. Гра хаактеризується системою правил, які визначають кількість учасників гри, їх можливі дії та розподіл виграшів в залежності від ситуації, що склалася в процесі проведення гри.

Під гравцем розуміють одного учасника або групу учасників, які мають спільні інтереси. \emph{Стратегією графця} називається набір правил, які вказують, який вибір варіанту дій він повинен зробити в залежності від ситуації, яка склалася в процесі проведення гри.

Стратегія є оптимальною, якщо в разі багато разового проведення гри, вона забезпечить гравцю максимально можливий середній виграш, або мінімально можливий середній програш.

\emph{Ходи} бувають особисі і випадкові.

\subsection*{Класифікація ігор}

В залежності від виду гри розробляються методи її розв'язування. Різні конфліктні ситуації призводять до різних видів ігор.

Основні напрямки класифікації:
\begin{slim_enumerate}
  \item Кількість гравців.
  \begin{slim_itemize}
    \item одно
    \item два
    \item багато
  \end{slim_itemize}
  \item Кількість стратегій.
  \begin{slim_itemize}
    \item скінчені
    \item нескінченні
  \end{slim_itemize}
  Якщо в грі кожен із гравців має скінченну кількість стратегій, то така гра називається скінченною. Якщо принаймні один із гравців маж нескінченну кількість стратегій, то така гра - нескінченна.
  \item Характер взаємозв'язків.
  \begin{slim_itemize}
    \item безкоаліційні
    \item кооперативні
  \end{slim_itemize}
  В безкоаліційній грі гравцям не дозволяється вступати в коаліцію, утворювати угоди і т.д.\\
  В кооперативній - коаліції визначені заздалегідь.
  \item Характер виграшів.
  \begin{slim_itemize}
    \item з нульовою сумою виграшів
    \item з ненульовою сумою виграшів
  \end{slim_itemize}
  В іграх з ненульовою сумою виграшів, загальний капітал гравців не змінюється, а перерозподіляється між гравцями в залежності від результату. Гра двох гравців з нульовою сумою виграшів є антагоністичною, оскільки виграш одного із гравців рівний програшу другого.
  \item Вигляд функції виграшів.
  \begin{slim_description}
    \item[матричні] гра двох гравців з нулевою сумою виграшу. В цьому випадку гра задається матрицею виграшів першого гравця. Виграш першого гравця = програшу другого.
    \item[біматричні] гра двох гравців, в якій виграші кожного із гравців задаються окремими матрицями.
    \item[неперервні] функція виграців - неперервна.
    \item[опуклі] функція виграшів - опукла.
    \item[сепарабельні] функція виграшів задається сумою добутків функцій від однієї змінної.
    \item[типу дуелей] характеризуються моментом вибору та імовірністями розподілу виграшу в залежності від часу, що пройшов від початку гри до моменту вибору.
  \end{slim_description}
  \item Кількість ходів.
  \begin{slim_itemize}
    \item однокрокові
    \item багатокрокові
  \end{slim_itemize}
  Матриця двох гравців з нульовою сумою виграшів є однокроковою. Кожен із гравців робить по одному ходу.
  \item Стан інформації
  \begin{slim_itemize}
    \item з повною інформацією
    \item з неповною інформацією
  \end{slim_itemize}
  Якщо в грі відомо про всі попередні вибари гравців, то така гра є - з повною інформацією.
\end{slim_enumerate}

Класифікація ігор є умовною, можлива й інша.

\clearpage

\section{Матрична гра двох гравців із нульовою сумою виграшів. Максимінні теореми.}

\clearpage

\section{Матрична гра двох гравців із нульовою сумою виграшів. Верхня та нижня ціни гри.}

\clearpage

\section{Оптимальні мішані стратегії та їх властивості.}

\clearpage

\section{Спрощення матричних ігор.}

\clearpage

\section{Гра порядку $2 \times 2$.}

\clearpage

\section{Ігри порядку $2 \times 2$, $m \times 2$.}

\clearpage

\section{Гра порядку $2 \times n$.}

\clearpage

\section{Гра порядку $m \times 2$.}

\clearpage

\section{Розв'язування матричних ігор шляхом зведення до задач лінійного програмування.}

\clearpage

\section{Поняття Біматричної гри. Умови рівноваги для біматричної гри.}

\clearpage

\section{Розв'язування біматричних ігор.}

\end{document}