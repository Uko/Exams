%%% Local Variables: 
%%% mode: latex
%%% TeX-master: t
%%% End: 

\documentclass[12pt,a4paper]{article}
\usepackage[ukrainian]{babel}
\usepackage[utf8]{inputenc}
\usepackage[T2A]{fontenc}
\usepackage[left=2cm,top=2cm,right=2cm,bottom=2cm,nohead,nofoot]{geometry}
\usepackage{setspace}
\usepackage{makecell}

\setcounter{tocdepth}{1}

\newcommand{\diagcell}[4]{\diaghead({#1},{#2}){easterr}{#4}{#3}}

\newenvironment{slim_enumerate}{
\begin{enumerate}
  \setlength{\itemsep}{1pt}
  \setlength{\parskip}{0pt}
  \setlength{\parsep}{0pt}}
{\end{enumerate}}

\newenvironment{slim_itemize}{
\begin{itemize}
  \setlength{\itemsep}{1pt}
  \setlength{\parskip}{0pt}
  \setlength{\parsep}{0pt}}
{\end{itemize}}

\newenvironment{slim_description}{
\begin{description}
  \setlength{\itemsep}{1pt}
  \setlength{\parskip}{0pt}
  \setlength{\parsep}{0pt}}
{\end{description}}

\begin{document}

\pretolerance=-1
\tolerance=6500

\pagestyle{empty}

\tableofcontents
\clearpage

\setstretch{1}
\fontsize{14pt}{6mm}\selectfont

\section{Задача про мінімальний каркас. Алгоритм Пріма.}

Нехай дано простий зв’язний зважений граф $G=(V,E)$ і вагова функція $d:E\rightarrow R$.

Потрібно знайти мінімальний каркас $A_s$ в заданому графі, починаючи з вершини $x_s$.

Алгоритм Пріма:
\begin{slim_enumerate}
  \item Нехай $T_s = \{x_s\}$ - множина вершин, з’єднаних ребрами, що входять в мінімальний каркас,\\
$A_s = \{\emptyset\}$ - множина ребер, що входять в каркас мінімальної довжини.
  \item Записати:\\
$\forall x_j\in$ Г$(x_s)$ $[\alpha_j=x_s, \beta_j=d(x_s,x_j)]$ (Г$(x_s)$ - суміжні до $x_s$ вершини)\\
$\forall x_j\notin$ Г$(x_s)$ $[0,\infty]$
  \item Вибрати $x_j^*$, де $\beta_j^*=\displaystyle\min_{x_j\notin T_s}\{\beta_j\}$,\\
$T_s=T_s\cup\{x_j^*\}$,\\
$A_s=A_s\cup\{(\alpha_j^*,x_j^*)\}$.\\
Якщо $|T_s|=n\Rightarrow$ кінець,\\
інакше ${\Rightarrow}$ Крок 4.
  \item $\forall x_j\notin T_s, x_j\in$ Г$(x_j^*), \beta_j>d(x_j^*,x_j)$ оновити мітки:\\
$\beta_j=d(x_j^*,x_j), \alpha_j=x_j^*$.\\
Перейти на Крок 3.
\end{slim_enumerate}

*місце на граф (малюнок-приклад)*

\clearpage

\section{Формулювання задач про найкоротший шлях. Знаходження найкоротшого шляху від заданої вершини (алгоритм Форда).}

Нехай маємо орієнтований граф $G=(V,E)$, дугам якого ставляться у відповідність ваги, що задаються матрицею $A=A_{ij}$. Ставляться такі задачі знаходження найкоротшого(х) шляху(ів):
\begin{slim_enumerate}
  \item від заданої початкової - до заданої кінцевої вершини графа;
  \item між заданою початковою вершиною графа та всіма іншими вершинами графа;
  \item між усіма парами вершин графа.
\end{slim_enumerate}
Задачі 1) і 2) розв’язують алгоритми Дейкстри (ваги $\geq 0$) і Форда (ваги довільні). Розглянемо другий.

Припустимо, що немає циклів з від’ємною довжиною. Навідміну від алгоритму Дейкстри, жодна мітка під час процесу не розглядаєтсья як остаточна.

Позначимо $l^k(x_i)$ - мітка вершини $x_i$ в кінці $k-1$ операції.
\begin{slim_enumerate}
  \item \emph{Присвоєння початкових значень}\\
Нехай $x_s$ - довільна початкова вершина,\\
покласти $S=$ Г$(x_s), k=1, l^1(x_s)=0;$\\
$\forall x_i \in$ Г$(x_s), l^1(x_i)=d(x_s,x_i)$\\
$\forall x_i \notin$ Г$(x_s), l^1(x_i)=\infty$
  \item \emph{Оновлення міток}\\
$\forall x_i \in$ Г$(S), (x_i \neq x_s)$ знайти її мітку наступним чином:\\
$T_i=$ Г$^{-1}(x_i) \cap S$,\\
$l^{k+1}(x_i)=\min\{l^k(x_i),\displaystyle\min_{x_j \in T_i}[l^k(x_j)+d(x_j,x_i)]\}$ (важливий порядок - дуги),\\
$\forall x_i \notin$ Г$(S): l^{k+1}(x_i)=l^k(x_i)$
  \item \emph{Перевірка на закінчення}
    \begin{slim_enumerate}
      \item $k \leq n-1$, якщо $\forall i$ $l^{k+1}(x_i)=l^k(x_i)$, то мітки рівні довжинам найкоротших шляхів. Кінець.
      \item $k<n-1$, якщо $\exists i$ $l^{k+1}(x_i) \neq l^k(x_i)$, то перейти до Кроку 4.
      \item $k=n-1$, якщо $\exists i$ $l^{k+1}(x_i) \neq l^k(x_i)$, то в графі присутній цикл від’ємної довжини і \emph{задача не має розв’язку}. Кінець.
    \end{slim_enumerate}
  \item \emph{Підготовка до наступної ітерації}\\
Оновити мітку наступним чином:
$$S=\{x_i:l^{k+1}(x_i) \neq l^k(x_i)\}$$
  \item Покласти $k=k+1$ і перейти до Кроку 2.
\end{slim_enumerate}

Коли довжини найкоротших шляхів будуть знайдені то самі шляхи отримаємо рекурсивно: $l(x_i')+d(x_i',x_i)=l(x_i), x_i'$ - вершина, що безпосередньо передує $x_i$ на шляху від $x_s$.

\clearpage

\section{Знаходження найкоротших шляхів між будь-якими вершинами графа (алгоритм Флойда).}

Цю задачу можна було б розв’язати методом багаторазового використання алгоритму Декстри чи Форда з послідовним перебором кожної вершини графа в ролі початкової, але це вимагало б великої обчислюваної роботи.

Більш ефективним методом розв’язування задачі 3) є алгоритм Флойда. Він застосовується до графів з довільними дугами, але не допускається наявність циклу від’ємної довжини.

В алгоритмі використовуються дві, оновлювані в його процесі, матриці: матриця ваг - $D$ і матриця попередніх вершин - $\Theta$. Прицьому, на $k$-й ітерації елементи матриці ваг $D_{(k)}=\{d_{ij}^{(k)}\}$ позначають найкоротший шлях між вершинами $x_i$ та $x_j$, який може складатись із внутрішніх (проміжних) вершин з множини перших $k$ вершин графу - $\{x_1, ..., x_k\}$ а елементи матриці $\Theta=\{\theta_{ij}^{(k)}\}$ позначають вершини, що безпосереньо передують вершинам $x_j$ у біжучому найкоротшому шляху від $x_i$ до $x_j$.

\emph{Внутрішня (проміжна) вершина - вершина графу, що не збігається з його початковою або кінцевою вершиною.}

\begin{slim_enumerate}
  \item \emph{Присвоєння початкових значень}\\
Нехай задано зважений граф $G=(V,E)$ i вагова функція $d:E \rightarrow R$.
$k=0$\\
$\forall (i,j) \in E d_{ij}^{(k)}=d(x_i,x_j)$ (є дуги)\\
$\forall (i,j) \notin E d_{ij}^{(k)}=\infty$ (дуги відсутні)\\
$d_{ii}^{(k)}=0$ (діагональні)\\
$\theta_{ij}^{(k)}=x_i$
  \item $k=k+1$
  \item $\forall i: i \neq k, d_{ik}^{(k-1)} \neq \infty, \forall j: j \neq k, d_{kj}^{(k-1)} \neq \infty, d_{ij}^{(k-1)}>d_{ik}^{(k-1)}+d_{kj}^{(k-1)}$ оновити матриці:\\
$d_{ij}^{(k)}=d_{ik}^{(k-1)}+d_{kj}^{(k-1)}$\\
$\theta_{ij}^{(k)}=\theta_{kj}^{(k-1)}$\\
Для всіх інших $i$ та $j$ переписати попередні елементи:\\
$d_{ij}^{(k)}=d_{ij}^{(k-1)}$,\\
$\theta_{ij}^{(k)}=\theta_{ij}^{(k-1)}$.
  \item 
    \begin{slim_enumerate}
      \item $d_{ii}^{(k)} < 0$ - в графі пристуній цикл від’ємної довжини, що містить вершину $x_i$ - розв’язку не існує. Кінець.
      \item $d_{ii}^{(k)} \geq 0, k=n$ - маємо розв’язок - матриця $D^{(n)}$ містить найкоротші шляхи між вершинами графу. Кінець.
      \item $d_{ii}^{(k)} \geq 0, k<n$ - перейти на Крок 2.
    \end{slim_enumerate}
\end{slim_enumerate}

\emph{Зауваження.} Якщо в початковій матриці $D^{(0)}$ усі діагональні елементи покласти рівними $\infty$ то $d_{ii}^{(n)}$ буде рівним вазі ланцюга що проходить через $x_i$.

\clearpage

\section{Формулювання транспортної задачі. Властивості транспортної задачі.}

Нехай маємо $m$ пунктів виробництва однорідного продукту (бази, склади, ...) з потужностями, відповідно, $a_i, i = \overline{1, m}$. Маємо $n$ пунктів споживання, відповідно, з потребами $b_j, j =\overline{1, n}$.

Задається матриця перевезень $С = \{c_{ij}\}$, де $c_{ij}$ - вартість перевезення одиниці продукту з $i$-того пункту виробництва в $j$-й пункт споживання.

Потрібно знайти такий набір $x_{ij} \geq 0, i = \overline{1, m}, j = \overline{1, n}$, де  $x_{ij}$ - кількість одиниць продукту, яка перевозиться з $і$-го пункту виробництва в $j$-й пункт споживання, щоб виконувались наступні умови: 
\begin{equation}  \sum_{j=1}^n x_{ij} = a_i, i = \overline{1, m}. \end{equation}
\begin{equation} \sum_{i=1}^m x_{ij} = b_j, j = \overline{1, n}.   \end{equation}
\begin{equation} \sum_{i=1}^m \sum_{j=1}^n c_{ij} x_{ij} \to \min  \end{equation}

Невідємний набір $x_{ij}$, який задовольняє (3.1), (3.2), називається \emph{планом задачі} або \emph{допустимим розвязком}. Той із планів, який надає мінімум в (3.3), називається \emph{оптимальним планом} або розвязком транспортної задачі.

\emph{Зауважимо}, що транспортна задача, поставлена в такій формі, називається "транспортною задачею за критерієм вартості".

Умова \begin{equation}  \sum_{i=1}^n a_i = \sum_{j=1}^m b_j  \end{equation} називається умовою \emph{балансу мас}.

Транспортну задачу зручно зображати таблицею:\\
\begin{tabular}{ | c | c | c | c | c | }
\hline
\diaghead(4,3){easterr}{$c_{1 2}$}{$x_{1 2}$} & \diaghead(4,3){easterr}{$c_{1 2}$}{$x_{1 2}$} & \thead{\vdots} & \diaghead(4,3){easterr}{$c_{1 n}$}{$x_{1 n}$} & \thead{$a_1$} \\
\hline
\diaghead(4,3){easterr}{$c_{2 1}$}{$x_{2 1}$} & \diaghead(4,3){easterr}{$c_{2 2}$}{$x_{2 2}$} & \thead{\vdots} & \diaghead(4,3){easterr}{$c_{2 n}$}{$x_{2 n}$} & \thead{$a_2$} \\
\hline
 \thead{$\cdots$} & \thead{$\cdots$} & \thead{$\ddots$} & \thead{$\cdots$} & \thead{$\cdots$} \\
\hline
\diaghead(4,3){easterr}{$c_{m 1}$}{$x_{m 1}$} & \diaghead(4,3){easterr}{$c_{m 2}$}{$x_{m 2}$} & \thead{\vdots} & \diaghead(4,3){easterr}{$c_{m n}$}{$x_{m n}$} & \thead{$a_m$} \\
\hline
\thead{$b_1$} & \thead{$b_2$} & \thead{\vdots} & \thead{$b_n$} & \thead{} \\
\hline
\end{tabular}

Матричний вигляд умов (3.1), (3.2) - матриця обмежень:\\
\begin{tabular}{ @{\hspace{1.4em}}l l }
$
\setlength{\arraycolsep}{0.27em}
\begin{array}{ccccccccccccc}
A_{1 1} & A_{1 2} & \dots & A_{1 n} & A_{2 1} & A_{2 2} & \dots & A_{2 n} & \dots & A_{m 1} & A_{m 2} & \dots & A_{m n} 
\end{array}$ &  \\
\multicolumn{2}{l}{
$\left(
 \begin{array}{ccccccccccccc}
1 & 1 & \dots & 1 & 0 & 0 & \dots & 0 & \dots & 0 & 0 & \dots & 0 \\
0 & 0 & \dots & 0 & 1 & 1 & \dots & 1 & \dots & 0 & 0 & \dots & 0 \\
\dots & \dots & \dots & \dots & \dots & \dots & \dots & \dots & \dots & \dots & \dots & \dots & \dots \\
0 & 0 & \dots & 0 & 0 & 0 & \dots & 0 & \dots & 1 & 1 & \dots & 1 \\
1 & 0 & \dots & 0 & 1 & 0 & \dots & 0 & \dots & 1 & 0 & \dots & 0 \\
\dots & \dots & \dots & \dots & \dots & \dots & \dots & \dots & \dots & \dots & \dots & \dots & \dots \\
0 & 0 & \dots & 1 & 0 & 0 & \dots & 1 & \dots & 0 & 0 & \dots & 1
\end{array}\right)
\left(\begin{array}{c}
x_{1 1} \\
x_{1 2} \\
\dots \\
x_{1 n} \\
x_{m 1} \\
\dots \\
x_{m n}
\end{array}
\right)
=
\left(\begin{array}{c}
a_1 \\
a_2 \\
\dots \\
a_m \\
b_1 \\
\dots \\
b_n
\end{array}
\right)$}
\end{tabular}

\subsection{Теорема 1}

Ранг матриці обмежень транспортної задачі рівний $r=m+n-1$.

{\bf Доведення:}

Оскільки сума перших рівнянь (3.1) рівна сумі наступних $n$ рівнянь (3.2), тобто виконується умова (3.4), то звідси випливає, що $r \leq m+n-1$.

Для того, щоб показати, що $r=m+n-1$ виділимо в матриці обмежень квадратну матрицю розмірності $m+n-1$ визначник якої нерівний нулю. Для цього можна взяти вектори:\\
\begin{tabular}{ @{\hspace{1em}}l l }
$
\setlength{\arraycolsep}{0.23em}
\begin{array}{cccccccc}
A_{1 n} & A_{2 n} & \dots & A_{m n} &  A_{1 1} & A_{1 2} & \dots & A_{1 n-1} 
\end{array}$ &  \\
\multicolumn{2}{l}{
$\left|
 \begin{array}{cccccccc}
1 & 0 & \dots & 0 & 1 & 1 & \dots & 1 \\
0 & 1 & \dots & 0 & 0 & 0 & \dots & 0 \\
\dots & \dots & \dots & \dots & \dots & \dots & \dots & \dots \\
0 & 0 & \dots & 1 & 0 & 0 & \dots & 0 \\
0 & 0 & \dots & 0 & 1 & 0 & \dots & 0 \\
0 & 0 & \dots & 0 & 0 & 1 & \dots & 0 \\
\dots & \dots & \dots & \dots & \dots & \dots & \dots & \dots \\
0 & 0 & \dots & 0 & 0 & 0 & \dots & 1 
\end{array}\right|$}
\end{tabular}
$\neq 0$

\subsection{Теорема  2}

Для розвязності транспортної задачі необхідно і достатньо, щоб виконувалась умова балансу мас $\sum_{i=1}^m a_i = \sum_{j=1}^n b_j$ .

{\bf Доведення:}

{\it Необхідність.} Нехай $ x_{ij}^*, i = \overline{1, m}, j = \overline{1, n}$ , - розвязок транспортної задачі.

Оскільки  $\sum_{j=1}^n x_{ij}^* = a_i,  i = \overline{1, m};  \sum_{i=1}^m  x_{ij}^* = b_j ,  j = \overline{1, n}$ , то отримаємо $\sum_{i=1}^n a_i = \sum_{j=1}^m b_j$.

{\it Достатність.} Нехай виконується умова балансу мас, покладемо  $x_{ij} = \frac{a_ib_j}{\sum_{i=1}^m a_i}$. Сумуючи це співвідношення по $j$, отримаємо:

$\sum_{j=1}^n x_{ij} = \sum_{j=1}^n \frac{a_ib_j}{\sum_{i=1}^m a_i} = a_i,  i = \overline{1, m}.$

$\sum_{i=1}^m x_{ij} = \sum_{i=1}^m \frac{a_ib_j}{\sum_{i=1}^m a_i} = b_j,  j = \overline{1, n}.$

\clearpage

\section{Відкрита та закрита моделі транспортної задачі.}

\clearpage

\section{Опорні плани транспортної задачі та їх властивості.}

План $\{x_{ij}\}_{m,n}$ транспортної задачі називають \emph{опорним планом}, якщо вектори $A_{ij}$ (з матриці обмежень), що відповідають додатним компонентам плану лінійно незалежні.

Оскільки із Т1 $\Rightarrow$ ранг матриці обмедень транспортної задачі $r=m+n-1$, то додатних компонент опорний план може мати не більше ніж $m+n-1$.

\begin{slim_itemize}
  \item Опорний план, який має рівно $m+n-1$ додатних компонент - невироджений.
  \item Опорний план, який має менше $m+n-1$ додатних компонент - вироджений.
\end{slim_itemize}

\emph{Базисом ОП} називається довільна система із $m+n-1$ лінійно незалежних векторів $A_{ij}$, яка містить усі вектори $A_{ij}$, що відповідають додатним компонентам плану.

Поставимо взаємовідповідність між клітинами транспонованої таблиці і векторами $A_{ij}$. Кожній клітині $(i,j) \leftrightarrow A_{ij}$.

Набір клітин $$(i_1,j_1),(i_1,j_2),(i_2,j_2),\dots,(i_s,j_1)$$або$$(i_1,j_1),(i_2,j_1),(i_2,j_2),\dots,(i_1,j_s)$$
називають \emph{ланцюжком}. Звідси видно, що два сусідні елементи ланцюжка лежать або в одному рядку або в одному стовпчику.

\emph{Зауважимо}, що кількість елементів замкненого ланцюжка завжди парна.

\dots табличка з прикладом ланцюжка \dots

Для того, щоб система векторів $P$ була лінійно незалежною необхідно і достатньо, щоб із елементів множини $I$ неможна було скласти замкнений ланцюжок.\\

\subsection{Теорема - Критерій опорності плану ТЗ}

Для того щоб план ТЗ був опорним необхідно і достатньо, щоб із клітин, які відповідають додатним перевезенням не можна було скласти замкнений ланцюг.

{\bf Наслідок}

Нехай $B$ - базис опорного плану ТЗ, тоді для довільної клітини $(k,l)$ можна побудувати єдиний ланцюг із елементів множини пар індексів $S=\{(i,s)\}$, що відповідають векторам $B$, який замикається на клітині $(k,l)$.

\clearpage

\section{Критерій лінійної незалежності системи векторів $A_{ij}$ умов транспортної задачі.}

Нехай $P$ - довільна система векторів $A_{ij}$ умов транспортної задачі, $I$ - множина пар індексів $(i,j)$, які відповідають векторам $A_{ij} \in P$.

{\bf Теорема:}

Для того, щоб система векторів $P$ була лінійно незалежною необхідно і достатньо, щоб із елементів множини $I$ неможна було скласти замкнений ланцюжок.

{\bf Доведення:}

{\it Необхідність.} $P$ - лінійно незалежна система, покажем, що неможна замкнути ланцюжок. {\it Від супротивного.} Припустимо, що із елементів множини $I$ можна скласти замкнений ланцюжок: $$(i_1,j_1), (i_1,j_2), (i_2,j_2), \dots, (i_s,j_1).$$ Звідси випливає, враховуючи вигляд векторів $A_i$, $$A_{{i_1},{j_1}}-A_{{i_1},{j_2}}+A_{{i_2},{j_2}}-\dots-A_{{i_s},{j_1}}=0,$$ а тому система векторів лінійно залежна, що суперечить вхідній умові.

{\it Достатність.} Припустимо, що замкнений ланцюжок не скласти. Покажемо, що система векторів лінійно незалежна. {\it Від супротивного.} Нехай вектори лінійно залежні. Звідси випливає, що $\exists \alpha_{ij} \neq 0, (i,j) \in I:$ $$\sum_{(i,j) \in I}\alpha_{ij}A_{ij} = 0.$$

Нехай $\alpha_{{i_1}{j_1}} \neq 0$, тоді: 
$$\sum_{(i,j) \in I}\alpha_{ij}A_{ij} = -\alpha_{{i_1}{j_1}}A_{{i_1}{j_1}};$$
$$I_1 = I\setminus\{(i_1,j_1)\}.$$

Компонента $i_1$ вектора в правій частині не рівна нулю, тому в лівій частині існує принаймі один вектор $A_{{i_1}{j_2}}: \alpha_{{i_1}{j_2}}\neq0$, тоді 
$$\sum_{(i,j){\in}I}\alpha_{ij}A_{ij} = -\alpha_{{i_1}{j_1}}A_{{i_1}{j_1}}-\alpha_{{i_1}{j_2}}A_{{i_1}{j_2}}.$$

Оскільки $j_1 \neq j_2$ і $m + j_2$ компонента парвої частини не рівна нулю, то знайдеться принаймі один вектор $A_{{i_2}{j_2}}: \alpha_{{i_2}{j_2}} \neq 0$, тоді $$\sum_{(i,j) \in I}\alpha_{ij}A_{ij} = -\alpha_{{i_1}{j_1}}A_{{i_1}{j_1}}-\alpha_{{i_1}{j_2}}A_{{i_1}{j_2}}-\alpha_{{i_2}{j_2}}A_{{i_2}{j_2}}.$$
і т.д.

Цей процес скінченний, оскільки всі вектори в лівій частині різні, то врезультаті приходимо до:
$$0 = -\alpha_{{i_1}{j_1}}A_{{i_1}{j_1}}-\alpha_{{i_2}{j_1}}A_{{i_2}{j_1}}-\dots-\alpha_{{i_k}{j_k-1}}A_{{i_k}{j_k-1}}, i_k=i_s, 1 \leq s \leq k-2;$$
$$0 = -\alpha_{{i_1}{j_1}}A_{{i_1}{j_1}}-\alpha_{{i_1}{j_2}}A_{{i_1}{j_2}}-\dots-\alpha_{{i_k}{j_{k+1}}}A_{{i_k}{j_{k+1}}}, j_{k+1}=j_l, l \leq 1 \leq k-1.$$

Тоді із елементів $(i_1,j_1), (i_1,j_2), \dots, (i_k,j_k)$ можна скласти замкнений ланцюг:
$$(i_s,j_s), (i_{s+1},j_s), \dots, (i_k = i_s, j_{k-1})$$
$$(i_l,j_l), (i_l,j_{l+1}), \dots, (i_k, j_{k+1} = j_l),$$
що суперечить нашому припущенню.

\clearpage

\section{Критерій розкладу довільного вектора $A_{kl}$ через систему лінійно незалежних векторів $A_{ij}$ умов транспортної задачі.}

Нехай система векторів $P$ - лінійно-незалежна, $A_{kl} \notin P$. $I$ - множина пар індексів $(i,j)$, які відповідають векторам $A_{ij} \in P$.

{\bf Теорема}

Вектор $A_{kl}$ можна виразити через вектори системи $P$ тоді і тільки тоді, коли із пар індексів $(i,j)$ можна скласти ланцюг, що замикається на клітині $(k,l)$.

{\bf Доведення:}

{\it Необхідність.} Припустимо, що вектор $A_{kl}$ можна виразити через вектори системи $P$, покажемо, що можливо утворити ланцюг із елементів множини $I$, що замикається на клітині $(k,l)$. з припущення випливає, що система $P'=P \cup \{A_{kl}\}$ - лінійно залежна. Тоді з критерію лінійної незалежності системи векторів-умов ТЗ випливає, що із пар індексів множини $I'=I \cup \{(k,l)\}$ можна побудувати ланцюжок. Причому цей ланцюг проходить через клітину $(k,l)$, так як у протилежному випадку - система векторів $P$ була б лінійно залежною, що суперечить умові.

{\it Достатність.} Візьмемо ланцюг $I'$, що замикається на клітині $(k,l)$ і покажемо, що вектор $A_{kl}$ можна виразити через $P$:
$$(k,j_1),(i_1,j_1),\dots,(i_s,l),(k,l).$$
Враховуючи вигляд векторів $A_{ij}$ можна записати:
$$A_{kj_1} - A_{i_1j_1} + \dots + A_{i_sl} - A_{kl} = 0.$$
Звідси видно, що $A_{kl}$ - лінійна комбінація векторів $A_{ij} \in P$.

\clearpage

\section{Побудова початкових опорних планів транспортної задачі. Метод мінімального елемента.}

При побудові початкового ОП методом Пн.-Зх. кута ми не враховували елементів матриці вартостей. Природно сподіватися, що коли враховувати елементи матриці вартостей, то отримаємо ОП кращий від попереднього, тобто затрати на перевезення будуть меншими. Таким методом є метод мінімального елементу.

Серед елементів матриці вартостей шукаємо мінімальний. Припустимо це елемент $c_{kl}$. Заповнення таблиці починаємо із клітини $(k,l)$, аналогічно як в попередньому методі.
$$x_{kl} = \min\{a_k,b_l\},$$
при цьому, або рядок, або стовпчик із наступного розгляду викреслюємо. Відповідно змінюємо або запас в $k-$му пункті постачання, або потреби $l-$го пункту споживання.

В матриці, що залишається знову шукаєм мінімальний елемент і т.д.

\begin{tabular}{ | c | c | c | c | c |}
\hline
\diagcell{4}{3}{8}{3}	&	\diagcell{4}{3}{}{5}	&	\diagcell{4}{3}{1}{4}	&	\diagcell{4}{3}{}{7}		&	9\\
\hline
\diagcell{4}{3}{}{7}	&	\diagcell{4}{3}{}{8}	&	\diagcell{4}{3}{5}{9}	&	\diagcell{4}{3}{10}{11}	&	15\\
\hline
\diagcell{4}{3}{}{4}	&	\diagcell{4}{3}{10}{6}	&	\diagcell{4}{3}{13}{8}	&	\diagcell{4}{3}{}{14}		&	23\\
\hline
8	&	10	&	19	&	10		&\\
\hline
\end{tabular}

\clearpage

\section{Побудова початкових опорних планів транспортної задачі. Метод Фогеля.}

В кожному рядку шукаємо мінімальний елемент і наступний за ним по величині. Різницю записуємо справа від рядка. Аналогічно поступаєво із стовпчиками, записуємо внизу кожного.

Серед отриманих різниць шукаємо максимальну і в стовпчику чи рядку, якому віповідає максимальна різниця шукаємо мінімальний елемент матриці вартостей.

Заповнення транспортної таблиці починаємо з отриманої клітини аналогічно як і в попередніх методах.

\begin{tabular}{ | c | c | c | c | c |}
\hline
\diagcell{4}{3}{}{3}	&	\diagcell{4}{3}{}{5}	&	\diagcell{4}{3}{9}{4}	&	\diagcell{4}{3}{}{7}		&	9\\
\hline
\diagcell{4}{3}{}{7}	&	\diagcell{4}{3}{}{8}	&	\diagcell{4}{3}{5}{9}	&	\diagcell{4}{3}{10}{11}	&	15\\
\hline
\diagcell{4}{3}{8}{4}	&	\diagcell{4}{3}{10}{6}	&	\diagcell{4}{3}{5}{8}	&	\diagcell{4}{3}{}{14}		&	23\\
\hline
8	&	10	&	19	&	10		&\\
\hline
\end{tabular}

\clearpage

\section{Двоїста задача. Умови оптимальності.}

ТЗ є задачею лінійного програмування, а тому можемо записати двоїсту до неї.

Запишемо вихідну ТЗ.

\begin{equation} \sum_{i=1}^m \sum_{j=1}^n c_{ij} x_{ij} \to \min \end{equation}
\begin{equation}  \sum_{j=1}^n x_{ij} = a_i, i = \overline{1, m}. \end{equation}
\begin{equation} \sum_{i=1}^m x_{ij} = b_j, j = \overline{1, n}. \end{equation}
\begin{equation} x_{ij} \geq 0, i = \overline{1, m}, j = \overline{1, n} \end{equation}

Кожному обмеженню (3.6) поставимо у відповідність змінну $u_i, i = \overline{1, m}$.

Кожному обмеженню (3.7) поставимо у відповідність змінну $v_j, j = \overline{1, n}$.

\begin{equation} L^* = \sum_{i=1}^m a_i u_i + \sum_{j=1}^n b_j v_j \to \max \end{equation}
\begin{equation} u_i+v_j \leq c_{ij}, i = \overline{1, m}, j = \overline{1, n} \end{equation}

Матриця обмежень:\\
\begin{tabular}{ @{\hspace{1.4em}}l l }
  \multicolumn{2}{l}{
    $\left(\begin{array}{c}
        $Транспонована матриця обмежень $[m*n \times m+n]\\
      \end{array}\right)
    \left(\begin{array}{c}
        u_1 \\
        \dots \\
        u_m \\
        v_1 \\
        \dots \\
        v_n
      \end{array}\right)
\leq
    \left(\begin{array}{c}
        c_{11} \\
        c_{12} \\
        \dots \\
        c_{mn}
      \end{array}\right)$}
\end{tabular}

Зауважимо, що змінні $u_i$ відповідають обмеженням (6), а обмеження (6) - відповідають рядкам таблиці ТЗ, тому змінні $u_i$ відповідають рядкам таблиці ТЗ. Аналогічно, змінні $v_j$ відповідають обмеженням (7), а (7) - відповідають стовпцям таблиці ТЗ, тому $v_j$ відповідають стовпцям таблиці ТЗ.

\subsection{Теорема - Умови оптимальності ТЗ}

План $x_{ij}$ ТЗ - оптимальний, тоді і тільки тоді, коли для будь-якого рядка $i$ та кожного стовпчика $j$ знайдуться такі числа-потенціали $u_1,\dots,u_m,v_1,\dots,v_n$, що задовільняють умови:
\begin{equation} u_i+v_j = c_{ij}, x_{ij}>0 \end{equation}
\begin{equation} u_i+v_j \leq c_{ij}, x_{ij}=0 \end{equation}

{\bf Доведення:}

{\it Достатність.} Умови (11) і (12) виконуються. Візьмемо довільний план $\tilde{x_{ij}}$, тоді:\\
$$ L(\tilde{x}_{ij}) = \sum_{i=1}^m \sum_{j=1}^n c_{ij} \tilde{x}_{ij} \geq
 \sum_{i=1}^m \sum_{j=1}^n (u_i+v_j) \tilde{x}_{ij} = 
 \sum_{i=1}^m \sum_{j=1}^n \tilde{x}_{ij} u_i + \sum_{i=1}^m \sum_{j=1}^n  \tilde{x}_{ij} v_j = $$
$$ = \sum_{i=1}^m a_i u_i + \sum_{j=1}^n b_j v_j =
 \sum_{i=1}^m \sum_{j=1}^n x_{ij} u_i + \sum_{i=1}^m \sum_{j=1}^n  x_{ij} v_j =
 \sum_{i=1}^m \sum_{j=1}^n (u_i+v_j) x_{ij} = $$
$$ = \sum_{i=1}^m \sum_{j=1}^n c_{ij} x_{ij} = L(x_{ij}). $$

Отже $x_{ij}$ - оптимальний план.

{\it Необхідність.} $x_{ij}$ - оптимальний план. Оскільки задача (5) - (8) (ТЗ) має розв’язок, то в силу 1 теореми двоїстості - двоїста задача також має розв’язок: $u^*_i, i = \overline{1,m}, v^*_j, j= \overline{1,n}$. Звідси випливає, що: $u^*_i+v^*_j \leq c_{ij}, i = \overline{1,m}, v^*_j, j= \overline{1,n}$ - виконується умова (12).

З 2 теоремою двоїстості $\Rightarrow (u^*_i + v^*_j - c_{ij})x_{ij}=0$. Звідси, якщо $x_{ij}>0$, то $u^*_i + v^*_j = c_{ij}$ - виконується умова (11).

\clearpage

\section{Метод потенціалів розв'язування транспортної задачі.}

\emph{(Базується на умові оптимальності плану ТЗ).}

\begin{slim_enumerate}
  \item \emph{Попередній крок.} Знайти початковий опороний план будь-яким із методів, наприклад методом „Пн-Зх кута“. Припустимо, що план - невироджений.

  \item \emph{Перевірка оптимальності.} Для клітин, які ввійшли в початковий опорний план і які будем називати „заповненими“, використовуючи умову $u_i + v_j = c_{ij}, x_{ij} > 0$, шукаємо числа потенціали $u_i, v_j$. Так як всього невідомих є $n+m$, а заповнених клітин - $m+n-1$, то довільній із змінних, наприклад $u_1$, присвоюємо $0$ і знаходимо всі решту.

Всі інші клітини, які називатимемо "вільними" перевіряєм на умову оптимальності - $u_i + v_j \leq c_{ij}, x_{ij} = 0$:
\begin{equation} \sigma_{ij} = c_{ij} - u_i - v_j \geq 0\end{equation}
Якщо всі вільні клітини задовільняють цю умову - то маємо оптимальний план, в іншому випадку - будемо його покращувати.

  \item \emph{Покращення плану.} Вибрати клітину з найбільшим порушенням умови, нехай це клітина $(k,l)$ - $\sigma_{kl} = \min \sigma_{ij}$.

\emph{Зауваження.} Можна вибирати будь-яку іншу, або першу, клітину з порушенням умови потимальності, але вибір клітини з найбільшим порушенням приводить до оптимального плану швидше.

Із заповнених клітин, будуємо ланцюжок, який замикається на клітині $(k,l)$. Позначаємо клітини почерзі знаками "$+$","$-$", причому клітину $(k,l)$ позначаємо знаком "+". Серед клітин із знаком "$-$" вибираєм клітину із найменшим значенням $x_{ij}$ (перевезенням). Віднімаємо це значення від значень в усіх клітинах із знаком "$-$" та додаємо до значень в клітинах із знаком "$+$". Отримали новий опорний план - повертаємся до перевірки його оптимальності.
\end{slim_enumerate}

\clearpage

\section{Метод диференціальних рент.}

\clearpage

\section{Транспортна задача за критерієм часу.}

\clearpage

\section{Задача про призначення.}

\clearpage

\section{Задача про максимальний потік. Теорема Форда-Фалкерсона.}

\clearpage

\section{Методи гілок і меж. Загальна схема.}

\clearpage

\section{Задача комівояжера.}

\clearpage

\section{Методи гілок і меж для задачі комівояжера.}

\clearpage

\section{Методи гілок і меж для задач лінійного цілочислового програмування.}

\clearpage


\section{Планування на мережах. Основні поняття та визначення.}

\clearpage

\section{Планування на мережах. Структура та правила побудови.}

\clearpage

\section{Основні поняття теорії ігор. Класифікація ігор.}

Гра є математичною моделлю реальної конфліктної ситуації. \emph{Теорія Ігор} - розділ математики, в якому досліджуються питання поведінки і виробляються оптимальні правила, стратегії поведінки для кожного із учасників гри. Гра хаактеризується системою правил, які визначають кількість учасників гри, їх можливі дії та розподіл виграшів в залежності від ситуації, що склалася в процесі проведення гри.

Під гравцем розуміють одного учасника або групу учасників, які мають спільні інтереси. \emph{Стратегією графця} називається набір правил, які вказують, який вибір варіанту дій він повинен зробити в залежності від ситуації, яка склалася в процесі проведення гри.

Стратегія є оптимальною, якщо в разі багато разового проведення гри, вона забезпечить гравцю максимально можливий середній виграш, або мінімально можливий середній програш.

\emph{Ходи} бувають особисі і випадкові.

\subsection*{Класифікація ігор}

В залежності від виду гри розробляються методи її розв'язування. Різні конфліктні ситуації призводять до різних видів ігор.

Основні напрямки класифікації:
\begin{slim_enumerate}
  \item Кількість гравців.
  \begin{slim_itemize}
    \item одно
    \item два
    \item багато
  \end{slim_itemize}
  \item Кількість стратегій.
  \begin{slim_itemize}
    \item скінчені
    \item нескінченні
  \end{slim_itemize}
  Якщо в грі кожен із гравців має скінченну кількість стратегій, то така гра називається скінченною. Якщо принаймні один із гравців маж нескінченну кількість стратегій, то така гра - нескінченна.
  \item Характер взаємозв'язків.
  \begin{slim_itemize}
    \item безкоаліційні
    \item кооперативні
  \end{slim_itemize}
  В безкоаліційній грі гравцям не дозволяється вступати в коаліцію, утворювати угоди і т.д.\\
  В кооперативній - коаліції визначені заздалегідь.
  \item Характер виграшів.
  \begin{slim_itemize}
    \item з нульовою сумою виграшів
    \item з ненульовою сумою виграшів
  \end{slim_itemize}
  В іграх з ненульовою сумою виграшів, загальний капітал гравців не змінюється, а перерозподіляється між гравцями в залежності від результату. Гра двох гравців з нульовою сумою виграшів є антагоністичною, оскільки виграш одного із гравців рівний програшу другого.
  \item Вигляд функції виграшів.
  \begin{slim_description}
    \item[матричні] гра двох гравців з нулевою сумою виграшу. В цьому випадку гра задається матрицею виграшів першого гравця. Виграш першого гравця = програшу другого.
    \item[біматричні] гра двох гравців, в якій виграші кожного із гравців задаються окремими матрицями.
    \item[неперервні] функція виграців - неперервна.
    \item[опуклі] функція виграшів - опукла.
    \item[сепарабельні] функція виграшів задається сумою добутків функцій від однієї змінної.
    \item[типу дуелей] характеризуються моментом вибору та імовірністями розподілу виграшу в залежності від часу, що пройшов від початку гри до моменту вибору.
  \end{slim_description}
  \item Кількість ходів.
  \begin{slim_itemize}
    \item однокрокові
    \item багатокрокові
  \end{slim_itemize}
  Матриця двох гравців з нульовою сумою виграшів є однокроковою. Кожен із гравців робить по одному ходу.
  \item Стан інформації
  \begin{slim_itemize}
    \item з повною інформацією
    \item з неповною інформацією
  \end{slim_itemize}
  Якщо в грі відомо про всі попередні вибари гравців, то така гра є - з повною інформацією.
\end{slim_enumerate}

Класифікація ігор є умовною, можлива й інша.

\clearpage

\section{Матрична гра двох гравців із нульовою сумою виграшів. Верхня та нижня ціни гри.}

\subsection*{ШО це ваще таке з реального життя}

\begin{description}
\item[Гра з двома пальцями] \hfill \\
Одночасно незалежно один від одного два гравці показують одина або два пальці і називають цифру 1 або 2 -- кількість пальців, яка на думку гравця показана противником. Якщо обидва вгадали чи не вгадали -- нічия (0-виграш кожному). Якщо один вгадав, а другий - ні, то переможець отримує виграч рівний сумі паліців показаній обидвома гравцями.

При моделюванні цієї ситуації для кожного гравця можливий стан буде задаватись парою: \{a, b\}, де a -- кількість палців, b -- передбачення кількості пальців противника.
Відповідно можем побудувати матрицю

\begin{tabular}{ r | c | c | c | c | }
         & 1, 1 & 1, 2 & 2, 1 & 2, 2 \\ \hline
  1, 1 & 0 & 2 & -3 & 0 \\ \hline
  1, 2 & -2 & 0 & 0 & 3 \\ \hline
  2, 1 & 3 & 0 & 0 & -4 \\ \hline
  2, 2 & 0 & -3 & 4 & 0 \\ \hline
\end{tabular}

\end{description}

\subsection*{Купу теорії}
Кожен із гравців має певну кількість стратегій. Перший - m стратегій, другий - n стратегій.

Задається матриця $A=\{a_{ij}\}_{m, n}$ - де $a_{ij}$ - виграш першого гравця, ща умови, що він вибирає свою $i$-ту стратегію, а другий - $j$-ту.

Кожен із гравців, незалежно один від одного, робить по одному ходу. Перший гравець обирає свою $i$-ту стратегію, а другий - $j$-ту. На цьому гра закінчується.

Якщо $a_{ij}>0$, то другий гравець платить першому виграш у розмірі $a_{ij}$ умовних одиниць.\\
Якщо $a_{ij}<0$, то перший гравець платить другому виграш у розмірі $|a_{ij}|$ умовних одиниць.

Стратегії $i=\overline{1, m}$ та $j=\overline{1, n}$, відповідних гравців називаються чистими стратегіями.

Гра задана, якщо задана матриця виграшів першого гравця.

Кожен з гравців зацікавлений в максимальному виграші. Перший гравець аналізуючи свою матрицю виграшів розуміє, що другий гравець буде вибирати свої найкращі стратегії. У зв'язку з цим, він визначає $\displaystyle \min_j a_{ij} = \alpha_i$ і вибирає таку стратегію
\begin{equation}
	\max_i \alpha_i = \max_i \min_j a_{ij} = \alpha
\end{equation}
Стратегія, яка забезпечить виграш $\alpha$, називається \emph{максимінна} стратегія. $\alpha$ -- чиста нижня ціна гри.

Аналогічно міркує другий гравець, і визначає
\[
	\min_j \max_i a_{ij} = \beta
\]
Стратегія, що забезпечує виграш $\beta$ називається \emph{мінімаксною}. $\beta$ -- чиста верхня ціна гри.

Чисті верхня і нижня ціни гри означають, що перший гравеь при застосуванні своїх чистис стратігій може забезпечити собі виграш $\ge\alpha$, а другий - може не допустити виграш першого гравця $>\beta$

Стратегії $i_0, j_0$, відповідно першого та другого гравця, які забезпечуюють $\alpha = \beta = v$ називається сідловою точкою матриці гри, а $v$ - \emph{чистою ціною гри}.
Стратегія $i_0, j_0$, що відповідає сідловій точці і сідловому елементу $v=a_{i_0j_0}$, називається розв'язком матричної гри.

Очевидно, що $a_{ij_0} \le a_{i_0j_0} \le a_{i_0j},\,i=\overline{1, m},\,j=\overline{1, n}$. Звідси слідує просте правило для відшукання сідлової точки. В кожному рядку матриці шукаємо мінімальний елемент і перевіряємо чи він є максимальнийм в своєму стовпчику.

Якщо сідлова точка існує, то кажуть, що матриця має розв'язок в чистих стратегіях.

\clearpage

\section{Матрична гра двох гравців із нульовою сумою виграшів. Максимінні теореми.}

Нехай $f(x, y), x \in A, y \in B$

\begin{description}

\item[Теорема 1] \hfill \\
Якщо існують $\displaystyle \alpha = \max_{x \in A} \min_{y \in B} f(x, y); \beta = \min_{y \in B} \max_{x \in A} f(x, y)$, то $\alpha \le \beta$

\item[Доведення] \hfill \\
Використовуючи означення максимуму та мінімуму можемо записати:
\[
\min_{y \in B}f(x, y) \le f(x, y) \le \max_{x \in A}f(x, y)
\]
\[
\min_{y \in B}f(x, y) \le \max_{x \in A}f(x, y)
\]
Оскільки нерівність справджується при всіх $x$, то при максимальному буде справджуватись теж:
\[
\max_{x \in A} \min_{y \in B}f(x, y) \le \max_{x \in A}f(x, y)
\]
Аналогічно з ігриком в правій частині:
\[
\max_{x \in A} \min_{y \in B}f(x, y) \le \min_{y \in B} \max_{x \in A}f(x, y)
\]
Що і треба було довести.
\end{description}

Точка $(x_0, y_0)$ називається \emph{сідловою}, якщо
\begin{equation}
f(x, y_0) \le f(x_0, y_0) \le f(x_0, y),\, x \in A,\, y \in B.
\end{equation}

Матрицю A в матричній грі можна розглянути як частковий випадок $f(x, y)$, якщо покласти $x=i, y=j, f(x, y) = a_{ij}$.

Тоді з теореми 1 отримуєм, що в матричній грі з матрицею A чиста нижня ціна не перевищує чистої верхньої: $\alpha \le \beta$.

\begin{description}

\item[Теорема 2] \hfill \\
$f(x, y)$ -- дійсна функція двох змінних, $x \in A, y \in B$ та існують
\[
\max_{x \in A} \min_{y \in B}f(x, y); \min_{y \in B} \max_{x \in A}f(x, y)
\]
Для того, щоб $\max\min = \min\max$ необхідно і достатньо, щоб існувала сідлова точка $(x_0, y_0)$. І якщо так точка існує, то
\begin{equation}
f(x_0, y_0)=\max_{x \in A} \min_{y \in B}f(x, y) = \min_{y \in B} \max_{x \in A}f(x, y)
\end{equation}

\item[Доведення] \hfill \\
(Достатність) Нехай $(x_0, y_0)$ -- сідлова точка, отже:
\[
f(x, y_0) \le f(x_0, y_0) \le f(x_0, y),\, x \in A,\, y \in B.
\]
звідси
\begin{equation}
\max_{x \in A} f(x, y_0) \le f(x_0, y_0) \le \min_{y \in B} f(x_0, y)
\end{equation}
\[
\min_{y \in B} \max_{x \in A} f(x, y) \le \max_{x \in A} f(x, y_0) \le f(x_0, y_0) \le \min_{y \in B} f(x_0, y) \le \max_{x \in A}  \min_{y \in B} f(x, y)
\]
Звідси
\[
\min_{y \in B} \max_{x \in A} f(x, y)  \le \max_{x \in A}  \min_{y \in B} f(x, y)
\]
За теоремою один остання нерівність перетворюється у рівність:
\[
\min_{y \in B} \max_{x \in A} f(x, y)  = \max_{x \in A}  \min_{y \in B} f(x, y)
\]

(Необхідність) 
\[\min_{y \in B} \max_{x \in A} f(x, y)  = \max_{x \in A}  \min_{y \in B} f(x, y)\]
\[\min_{y \in B} \left(\max_{x \in A} f(x, y)\right)  = \max_{x \in A}  \left(\min_{y \in B} f(x, y)\right)\]
\[\min_{y \in B} \left(\max_{x \in A} f(x, y)\right)  = \max_{x \in A} f(x, y_0)\]
\[\max_{x \in A} \left(\min_{y \in B} f(x, y)\right)= \min_{y \in B} f(x_0, y)\]

Покажемо, що $(x_0, y_0)$ -- сідлова точка.
\begin{eqnarray}
\min_{y \in B} f(x_0, y)=\max_{x \in A} f(x, y_0)\\
\min_{y \in B} f(x_0, y) \le f(x_0, y_0)
\end{eqnarray}

Із (5), враховуючи (6), отримаєм:
\[\max_{x \in A} f(x, y_0) \le f(x_0, y_0) \]. Тим самим ми довели ліву нерівність в означені сідлової точки (2). Аналогічно доведеться права нерівність. Отже $(x_0, y_0)$ -- сідлова точка. І враховуючи вихідну умову, отримуємо (3).

\end{description}

\clearpage

\section{Оптимальні мішані стратегії та їх властивості.}

\clearpage

\section{Спрощення матричних ігор.}

\clearpage

\section{Гра порядку $2 \times 2$.}

\clearpage

\section{Гра порядку $2 \times n$.}

\clearpage

\section{Гра порядку $m \times 2$.}

\clearpage

\section{Розв'язування матричних ігор шляхом зведення до задач лінійного програмування.}

\clearpage

\section{Поняття Біматричної гри. Умови рівноваги для біматричної гри.}

\clearpage

\section{Розв'язування біматричних ігор.}

\end{document}