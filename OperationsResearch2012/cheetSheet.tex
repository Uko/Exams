%%% Local Variables: 
%%% mode: latex
%%% TeX-master: t
%%% End: 

\documentclass[12pt,a4paper]{article}
\usepackage[ukrainian]{babel}
\usepackage[utf8]{inputenc}
\usepackage[T2A]{fontenc}
\usepackage[left=2cm,top=2cm,right=2cm,bottom=2cm,nohead,nofoot]{geometry}
\usepackage{setspace}

\setcounter{tocdepth}{1}

\newenvironment{slim_enumerate}{
\begin{enumerate}
  \setlength{\itemsep}{1pt}
  \setlength{\parskip}{0pt}
  \setlength{\parsep}{0pt}}
{\end{enumerate}}

\newenvironment{slim_itemize}{
\begin{itemize}
  \setlength{\itemsep}{1pt}
  \setlength{\parskip}{0pt}
  \setlength{\parsep}{0pt}}
{\end{itemize}}

\newenvironment{slim_description}{
\begin{description}
  \setlength{\itemsep}{1pt}
  \setlength{\parskip}{0pt}
  \setlength{\parsep}{0pt}}
{\end{description}}

\begin{document}

\pretolerance=-1
\tolerance=6500

\pagestyle{empty}

\tableofcontents
\clearpage

\setstretch{1}
\fontsize{14pt}{6mm}\selectfont

\section{Задача про мінімальний каркас. Алгоритм Пріма.}

\clearpage

\section{Формулювання задач про найкоротший шлях. Знаходження найкоротшого шляху від заданої вершини (алгоритм Форда).}

\clearpage

\section{Знаходження найкоротших шляхів між будь-якими вершинами графа (алгоритм Флойда).}

\clearpage

\section{Задача про призначення.}

\clearpage

\section{Задача про максимальний потік. Теорема Форда-Фалкерсона.}

\clearpage

\section{Планування на мережах. Основні поняття та визначення.}

\clearpage

\section{Планування на мережах. Структура та правила побудови.}

\clearpage

\section{Формулювання транспортної задачі. Властивості транспортної задачі.}

\clearpage

\section{Опорні плани транспортної задачі та їх властивості.}

\clearpage

\section{Побудова початкових опорних планів транспортної задачі. Метод мінімального елемента.}

\clearpage

\section{Відкрита та закрита моделі транспортної задачі.}

\clearpage

\section{Критерій лінійної незалежності системи векторів $A_{ij}$ умов транспортної задачі.}

\clearpage

\section{Побудова початкових опорних планів транспортної задачі. Метод Фогеля.}

\clearpage

\section{Критерій розкладу довільного вектора $A_{kl}$ через систему лінійно незалежних векторів $A_{ij}$ умов транспортної задачі.}

\clearpage

\section{Двоїста задача. Умови оптимальності.}

\clearpage

\section{Метод потенціалів розв'язування транспортної задачі.}

\clearpage

\section{Метод диференціальних рент.}

\clearpage

\section{Транспортна задача за критерієм часу.}

\clearpage

\section{Методи гілок і меж. Загальна схема.}

\clearpage

\section{Задача комівояжера.}

\clearpage

\section{Методи гілок і меж для задачі комівояжера.}

\clearpage

\section{Методи гілок і меж для задач лінійного цілочислового програмування.}

\clearpage

\section{Основні поняття теорії ігор. Класифікація ігор.}

Гра є математичною моделлю реальної конфліктної ситуації. \emph{Теорія Ігор} - розділ математики, в якому досліджуються питання поведінки і виробляються оптимальні правила, стратегії поведінки для кожного із учасників гри. Гра хаактеризується системою правил, які визначають кількість учасників гри, їх можливі дії та розподіл виграшів в залежності від ситуації, що склалася в процесі проведення гри.

Під гравцем розуміють одного учасника або групу учасників, які мають спільні інтереси. \emph{Стратегією графця} називається набір правил, які вказують, який вибір варіанту дій він повинен зробити в залежності від ситуації, яка склалася в процесі проведення гри.

Стратегія є потимальною, якщо в разі багато разового проведення гри, вона забезпечить гравцю максимально можливий середній виграш, або мінімально можливий середній програш.

\emph{Ходи} бувають особисі і випадкові.

\subsection*{Класифікація ігор}

В залежності від виду гри розробляються методи її розв'язування. Різні конфліктні ситуації призводять до різних видів ігор.

Основні напрямки класифікації:
\begin{slim_enumerate}
  \item Кількість гравців.
  \begin{slim_itemize}
    \item одно
    \item два
    \item багато
  \end{slim_itemize}
  \item Кількість стратегій.
  \begin{slim_itemize}
    \item скінчені
    \item нескінченні
  \end{slim_itemize}
  Якщо в грі кожен із гравців має скінченну кількість стратегій, то така гра називається скінченною. Якщо принаймні один із гравців маж нескінченну кількість стратегій, то така гра - нескінченна.
  \item Характер взаємозв'язків.
  \begin{slim_itemize}
    \item безкоаліційні
    \item кооперативні
  \end{slim_itemize}
  В безкоаліційній грі гравцям не дозволяється вступати в коаліцію, утворювати угоди і т.д.\\
  В кооперативній - коаліції визначені заздалегідь.
  \item Характер виграшів.
  \begin{slim_itemize}
    \item з нульовою сумою виграшів
    \item з ненульовою сумою виграшів
  \end{slim_itemize}
  В іграх з ненульовою сумою виграшів, загальний капітал гравців не змінюється, а перерозподіляється між гравцями в залежності від результату. Гра двох гравців з нульовою сумою виграшів є антагоністичною, оскільки виграш одного із гравців рівний програшу другого.
  \item Вигляд функції виграшів.
  \begin{slim_description}
    \item[матричні] гра двох гравців з нулевою сумою виграшу. В цьому випадку гра задається матрицею виграшів першого гравця. Виграш першого гравця = програшу другого.
    \item[біматричні] гра двох гравців, в якій виграші кожного із гравців задаються окремими матрицями.
    \item[неперервні] функція виграців - неперервна.
    \item[опуклі] функція виграшів - опукла.
    \item[сепарабельні] функція виграшів задається сумою добутків функцій від однієї змінної.
    \item[типу дуелей] характеризуються моментом вибору та імовірністями розподілу виграшу в залежності від часу, що пройшов від початку гри до моменту вибору.
  \end{slim_description}
  \item Кількість ходів.
  \begin{slim_itemize}
    \item однокрокові
    \item багатокрокові
  \end{slim_itemize}
  Матриця двох гравців з нульовою сумою виграшів є однокроковою. Кожен із гравців робить по одному ходу.
  \item Стан інформації
  \begin{slim_itemize}
    \item з повною інформацією
    \item з неповною інформацією
  \end{slim_itemize}
  Якщо в грі відомо про всі попередні вибари гравців, то така гра є - з повною інформацією.
\end{slim_enumerate}

Класифікація ігор є умовною, можлива й інша.

\clearpage

\section{Матрична гра двох гравців із нульовою сумою виграшів. Максимінні теореми.}

\subsection*{ШО це ваще таке з реального життя}

\begin{description}
\item[Гра з двома пальцями] \hfill \\
Одночасно незалежно один від одного два гравці показують одина або два пальці і називають цифру 1 або 2 -- кількість пальців, яка на думку гравця показана противником. Якщо обидва вгадали чи не вгадали -- нічия (0-виграш кожному). Якщо один вгадав, а другий - ні, то переможець отримує виграч рівний сумі паліців показаній обидвома гравцями.

При моделюванні цієї ситуації для кожного гравця можливий стан буде задаватись парою: \{a, b\}, де a -- кількість палців, b -- передбачення кількості пальців противника.
Відповідно можем побудувати матрицю

\begin{tabular}{ r | c | c | c | c | }
         & 1, 1 & 1, 2 & 2, 1 & 2, 2 \\ \hline
  1, 1 & 0 & 2 & -3 & 0 \\ \hline
  1, 2 & -2 & 0 & 0 & 3 \\ \hline
  2, 1 & 3 & 0 & 0 & -4 \\ \hline
  2, 2 & 0 & -3 & 4 & 0 \\ \hline
\end{tabular}

\end{description}

\subsection*{Купу теорії}
Кожен із гравців має певну кількість стратегій. Перший - m стратегій, другий - n стратегій.

Задається матриця $A=\{a_{ij}\}_{m, n}$ - де $a_{ij}$ - виграш першого гравця, ща умови, що він вибирає свою $i$-ту стратегію, а другий - $j$-ту.

Кожен із гравців, незалежно один від одного, робить по одному ходу. Перший гравець обирає свою $i$-ту стратегію, а другий - $j$-ту. На цьому гра закінчується.

Якщо $a_{ij}>0$, то другий гравець платить першому виграш у розмірі $a_{ij}$ умовних одиниць.\\
Якщо $a_{ij}<0$, то перший гравець платить другому виграш у розмірі $|a_{ij}|$ умовних одиниць.

Стратегії $i=\overline{1, m}$ та $j=\overline{1, n}$, відповідних гравців називаються чистими стратегіями.

Гра задана, якщо задана матриця виграшів першого гравця.

Кожен з гравців зацікавлений в максимальному виграші. Перший гравець аналізуючи свою матрицю виграшів розуміє, що другий гравець буде вибирати свої найкращі стратегії. У зв'язку з цим, він визначає $\displaystyle \min_j a_{ij} = \alpha_i$ і вибирає таку стратегію
\begin{equation}
	\max_i \alpha_i = \max_i \min_j a_{ij} = \alpha
\end{equation}
Стратегія, яка забезпечить виграш $\alpha$, називається \emph{максимінна} стратегія. $\alpha$ -- чиста нижня ціна гри.

Аналогічно міркує другий гравець, і визначає
\[
	\min_j \max_i a_{ij} = \beta
\]
Стратегія, що забезпечує виграш $\beta$ називається \emph{мінімаксною}. $\beta$ -- чиста верхня ціна гри.

Чисті верхня і нижня ціни гри означають, що перший гравеь при застосуванні своїх чистис стратігій може забезпечити собі виграш $\ge\alpha$, а другий - може не допустити виграш першого гравця $>\beta$

Стратегії $i_0, j_0$, відповідно першого та другого гравця, які забезпечуюють $\alpha = \beta = v$ називається сідловою точкою матриці гри, а $v$ - \emph{чистою ціною гри}.
Стратегія $i_0, j_0$, що відповідає сідловій точці і сідловому елементу $v=a_{i_0j_0}$, називається розв'язком матричної гри.

Очевидно, що $a_{ij_0} \le a_{i_0j_0} \le a_{i_0j},\,i=\overline{1, m},\,j=\overline{1, n}$. Звідси слідує просте правило для відшукання сідлової точки. В кожному рядку матриці шукаємо мінімальний елемент і перевіряємо чи він є максимальнийм в своєму стовпчику.

Якщо сідлова точка існує, то кажуть, що матриця має розв'язок в чистих стратегіях.

\subsection*{Максимінні теореми}

Нехай $f(x, y), x \in A, y \in B$

\begin{description}

\item[Теорема 1] \hfill \\
Якщо існують $\displaystyle \alpha = \max_{x \in A} \min_{y \in B} f(x, y); \beta = \min_{y \in B} \max_{x \in A} f(x, y)$, то $\alpha \le \beta$

\item[Доведення] \hfill \\
Використовуючи означення максимуму та мінімуму можемо записати:
\[
\min_{y \in B}f(x, y) \le f(x, y) \le \max_{x \in A}f(x, y)
\]
\[
\min_{y \in B}f(x, y) \le \max_{x \in A}f(x, y)
\]
Оскільки нерівність справджується при всіх $x$, то при максимальному буде справджуватись теж:
\[
\max_{x \in A} \min_{y \in B}f(x, y) \le \max_{x \in A}f(x, y)
\]
Аналогічно з ігриком в правій частині:
\[
\max_{x \in A} \min_{y \in B}f(x, y) \le \min_{y \in B} \max_{x \in A}f(x, y)
\]
Що і треба було довести.
\end{description}

Точка $(x_0, y_0)$ називається \emph{сідловою}, якщо
\begin{equation}
f(x, y_0) \le f(x_0, y_0) \le f(x_0, y),\, x \in A,\, y \in B.
\end{equation}

Матрицю A в матричній грі можна розглянути як частковий випадок $f(x, y)$, якщо покласти $x=i, y=j, f(x, y) = a_{ij}$.

Тоді з теореми 1 отримуєм, що в матричній грі з матрицею A чиста нижня ціна не перевищує чистої верхньої: $\alpha \le \beta$.

\begin{description}

\item[Теорема 2] \hfill \\
$f(x, y)$ -- дійсна функція двох змінних, $x \in A, y \in B$ та існують
\[
\max_{x \in A} \min_{y \in B}f(x, y); \min_{y \in B} \max_{x \in A}f(x, y)
\]
Для того, щоб $\max\min = \min\max$ необхідно і достатньо, щоб існувала сідлова точка $(x_0, y_0)$. І якщо так точка існує, то
\begin{equation}
f(x_0, y_0)=\max_{x \in A} \min_{y \in B}f(x, y) = \min_{y \in B} \max_{x \in A}f(x, y)
\end{equation}

\item[Доведення] \hfill \\
(Достатність) Нехай $(x_0, y_0)$ -- сідлова точка, отже:
\[
f(x, y_0) \le f(x_0, y_0) \le f(x_0, y),\, x \in A,\, y \in B.
\]
звідси
\begin{equation}
\max_{x \in A} f(x, y_0) \le f(x_0, y_0) \le \min_{y \in B} f(x_0, y)
\end{equation}
\[
\min_{y \in B} \max_{x \in A} f(x, y) \le \max_{x \in A} f(x, y_0) \le f(x_0, y_0) \le \min_{y \in B} f(x_0, y) \le \max_{x \in A}  \min_{y \in B} f(x, y)
\]
Звідси
\[
\min_{y \in B} \max_{x \in A} f(x, y)  \le \max_{x \in A}  \min_{y \in B} f(x, y)
\]
За теоремою один остання нерівність перетворюється у рівність:
\[
\min_{y \in B} \max_{x \in A} f(x, y)  = \max_{x \in A}  \min_{y \in B} f(x, y)
\]

(Необхідність) 
\[\min_{y \in B} \max_{x \in A} f(x, y)  = \max_{x \in A}  \min_{y \in B} f(x, y)\]
\[\min_{y \in B} \left(\max_{x \in A} f(x, y)\right)  = \max_{x \in A}  \left(\min_{y \in B} f(x, y)\right)\]
\[\min_{y \in B} \left(\max_{x \in A} f(x, y)\right)  = \max_{x \in A} f(x, y_0)\]
\[\max_{x \in A} \left(\min_{y \in B} f(x, y)\right)= \min_{y \in B} f(x_0, y)\]

Покажемо, що $(x_0, y_0)$ -- сідлова точка.
\begin{eqnarray}
\min_{y \in B} f(x_0, y)=\max_{x \in A} f(x, y_0)\\
\min_{y \in B} f(x_0, y) \le f(x_0, y_0)
\end{eqnarray}

Із (5), враховуючи (6), отримаєм:
\[\max_{x \in A} f(x, y_0) \le f(x_0, y_0) \]. Тим самим ми довели ліву нерівність в означені сідлової точки (2). Аналогічно доведеться права нерівність. Отже $(x_0, y_0)$ -- сідлова точка. І враховуючи вихідну умову, отримуємо (3).

\end{description}

\clearpage

\section{Матрична гра двох гравців із нульовою сумою виграшів. Верхня та нижня ціни гри.}

\clearpage

\section{Оптимальні мішані стратегії та їх властивості.}

\clearpage

\section{Спрощення матричних ігор.}

\clearpage

\section{Гра порядку $2 \times 2$.}

\clearpage

\section{Ігри порядку $2 \times 2$, $m \times 2$.}

\clearpage

\section{Гра порядку $2 \times n$.}

\clearpage

\section{Гра порядку $m \times 2$.}

\clearpage

\section{Розв'язування матричних ігор шляхом зведення до задач лінійного програмування.}

\clearpage

\section{Поняття Біматричної гри. Умови рівноваги для біматричної гри.}

\clearpage

\section{Розв'язування біматричних ігор.}

\end{document}