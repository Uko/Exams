%%% Local Variables: 
%%% mode: latex
%%% TeX-master: t
%%% End: 

\documentclass[12pt,a4paper]{article}
\usepackage[ukrainian]{babel}
\usepackage[utf8]{inputenc}
\usepackage[T2A]{fontenc}
\usepackage[left=2cm,top=2cm,right=2cm,bottom=2cm,nohead,nofoot]{geometry}
\usepackage{setspace}

\begin{document}

\pretolerance=-1
\tolerance=6500
\setstretch{1.5}
\fontsize{14pt}{6mm}\selectfont

\pagestyle{empty}

\tableofcontents
\clearpage

\section{Задача про мінімальний каркас. Алгоритм Пріма.}

\clearpage

\section{Формулювання задач про найкоротший шлях. Знаходження найкоротшого шляху від заданої вершини (алгоритм Форда).}

\clearpage

\section{Знаходження найкоротших шляхів між будь-якими вершинами графа (алгоритм Флойда).}

\clearpage

\section{Задача про призначення.}

\clearpage

\section{Задача про максимальний потік. Теорема Форда-Фалкерсона.}

\clearpage

\section{Планування на мережах. Основні поняття та визначення.}

\clearpage

\section{Планування на мережах. Структура та правила побудови.}

\clearpage

\section{Формулювання транспортної задачі. Властивості транспортної задачі.}

\clearpage

\section{Опорні плани транспортної задачі та їх властивості.}

\clearpage

\section{Побудова початкових опорних планів транспортної задачі. Метод мінімального елемента.}

\clearpage

\section{Відкрита та закрита моделі транспортної задачі.}

\clearpage

\section{Критерій лінійної незалежності системи векторів $A_{ij}$ умов транспортної задачі.}

\clearpage

\section{Побудова початкових опорних планів транспортної задачі. Метод Фогеля.}

\clearpage

\section{Критерій розкладу довільного вектора $A_{kl}$ через систему лінійно незалежних векторів $A_{ij}$ умов транспортної задачі.}

\clearpage

\section{Двоїста задача. Умови оптимальності.}

\clearpage

\section{Метод потенціалів розв'язування транспортної задачі.}

\clearpage

\section{Метод диференціальних рент.}

\clearpage

\section{Транспортна задача за критерієм часу.}

\clearpage

\section{Методи гілок і меж. Загальна схема.}

\clearpage

\section{Задача комівояжера.}

\clearpage

\section{Методи гілок і меж для задачі комівояжера.}

\clearpage

\section{Методи гілок і меж для задач лінійного цілочислового програмування.}

\clearpage

\section{Основні поняття теорії ігор. Класифікація ігор.}

\clearpage

\section{Матрична гра двох гравців із нульовою сумою виграшів. Максимінні теореми.}

\clearpage

\section{Матрична гра двох гравців із нульовою сумою виграшів. Верхня та нижня ціни гри.}

\clearpage

\section{Оптимальні мішані стратегії та їх властивості.}

\clearpage

\section{Спрощення матричних ігор.}

\clearpage

\section{Гра порядку $2 \times 2$.}

\clearpage

\section{Ігри порядку $2 \times 2$, $m \times 2$.}

\clearpage

\section{Гра порядку $2 \times n$.}

\clearpage

\section{Гра порядку $m \times 2$.}

\clearpage

\section{Розв'язування матричних ігор шляхом зведення до задач лінійного програмування.}

\clearpage

\section{Поняття Біматричної гри. Умови рівноваги для біматричної гри.}

\clearpage

\section{Розв'язування біматричних ігор.}

\end{document}