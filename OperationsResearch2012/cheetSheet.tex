%%% Local Variables: 
%%% mode: latex
%%% TeX-master: t
%%% End: 

\documentclass[12pt,a4paper]{report}
\usepackage[ukrainian]{babel}
\usepackage[utf8]{inputenc}
\usepackage[T2A]{fontenc}
\usepackage[left=2cm,top=2cm,right=2cm,bottom=2cm,nohead,nofoot]{geometry}
\usepackage{setspace}
\usepackage{makecell}

\makeatletter
\renewcommand{\@makechapterhead}[1]{%
\vspace*{0 pt}%
{\setlength{\parindent}{0pt} \raggedright \normalfont
\bfseries\Large\thechapter.\ #1
\par\nobreak\vspace{10 pt}}}
\makeatother

\setcounter{tocdepth}{1}

\newcommand{\diagcell}[4]{\diaghead({#1},{#2}){easterr}{#4}{#3}}

\newenvironment{slim_enumerate}{
\begin{enumerate}
  \setlength{\itemsep}{1pt}
  \setlength{\parskip}{0pt}
  \setlength{\parsep}{0pt}}
{\end{enumerate}}

\newenvironment{slim_itemize}{
\begin{itemize}
  \setlength{\itemsep}{1pt}
  \setlength{\parskip}{0pt}
  \setlength{\parsep}{0pt}}
{\end{itemize}}

\newenvironment{slim_description}{
\begin{description}
  \setlength{\itemsep}{1pt}
  \setlength{\parskip}{0pt}
  \setlength{\parsep}{0pt}}
{\end{description}}

\begin{document}

\pretolerance=-1
\tolerance=6500

\pagestyle{empty}

\tableofcontents
\clearpage

\setstretch{1}
\fontsize{14pt}{6mm}\selectfont

\chapter{Задача про мінімальний каркас. Алгоритм Пріма.}

Нехай дано простий зв’язний зважений граф $G=(V,E)$ і вагова функція $d:E\rightarrow R$.

Потрібно знайти мінімальний каркас $A_s$ в заданому графі, починаючи з вершини $x_s$.

Алгоритм Пріма:
\begin{slim_enumerate}
  \item Нехай $T_s = \{x_s\}$ - множина вершин, з’єднаних ребрами, що входять в мінімальний каркас,\\
$A_s = \{\emptyset\}$ - множина ребер, що входять в каркас мінімальної довжини.
  \item Записати:\\
$\forall x_j\in$ Г$(x_s)$ $[\alpha_j=x_s, \beta_j=d(x_s,x_j)]$ (Г$(x_s)$ - суміжні до $x_s$ вершини)\\
$\forall x_j\notin$ Г$(x_s)$ $[0,\infty]$
  \item Вибрати $x_j^*$, де $\beta_j^*=\displaystyle\min_{x_j\notin T_s}\{\beta_j\}$,\\
$T_s=T_s\cup\{x_j^*\}$,\\
$A_s=A_s\cup\{(\alpha_j^*,x_j^*)\}$.\\
Якщо $|T_s|=n\Rightarrow$ кінець,\\
інакше ${\Rightarrow}$ Крок 4.
  \item $\forall x_j\notin T_s, x_j\in$ Г$(x_j^*), \beta_j>d(x_j^*,x_j)$ оновити мітки:\\
$\beta_j=d(x_j^*,x_j), \alpha_j=x_j^*$.\\
Перейти на Крок 3.
\end{slim_enumerate}

*місце на граф (малюнок-приклад)*

\clearpage

\chapter{Формулювання задач про найкоротший шлях. Знаходження найкоротшого шляху від заданої вершини (алгоритм Форда).}

Нехай маємо орієнтований граф $G=(V,E)$, дугам якого ставляться у відповідність ваги, що задаються матрицею $A=A_{ij}$. Ставляться такі задачі знаходження найкоротшого(х) шляху(ів):
\begin{slim_enumerate}
  \item від заданої початкової - до заданої кінцевої вершини графа;
  \item між заданою початковою вершиною графа та всіма іншими вершинами графа;
  \item між усіма парами вершин графа.
\end{slim_enumerate}
Задачі 1) і 2) розв’язують алгоритми Дейкстри (ваги $\geq 0$) і Форда (ваги довільні). Розглянемо другий.

Припустимо, що немає циклів з від’ємною довжиною. Навідміну від алгоритму Дейкстри, жодна мітка під час процесу не розглядаєтсья як остаточна.

Позначимо $l^k(x_i)$ - мітка вершини $x_i$ в кінці $k-1$ операції.
\begin{slim_enumerate}
  \item \emph{Присвоєння початкових значень}\\
Нехай $x_s$ - довільна початкова вершина,\\
покласти $S=$ Г$(x_s), k=1, l^1(x_s)=0;$\\
$\forall x_i \in$ Г$(x_s), l^1(x_i)=d(x_s,x_i)$\\
$\forall x_i \notin$ Г$(x_s), l^1(x_i)=\infty$
  \item \emph{Оновлення міток}\\
$\forall x_i \in$ Г$(S), (x_i \neq x_s)$ знайти її мітку наступним чином:\\
$T_i=$ Г$^{-1}(x_i) \cap S$,\\
$l^{k+1}(x_i)=\min\{l^k(x_i),\displaystyle\min_{x_j \in T_i}[l^k(x_j)+d(x_j,x_i)]\}$ (важливий порядок - дуги),\\
$\forall x_i \notin$ Г$(S): l^{k+1}(x_i)=l^k(x_i)$
  \item \emph{Перевірка на закінчення}
    \begin{slim_enumerate}
      \item $k \leq n-1$, якщо $\forall i$ $l^{k+1}(x_i)=l^k(x_i)$, то мітки рівні довжинам найкоротших шляхів. Кінець.
      \item $k<n-1$, якщо $\exists i$ $l^{k+1}(x_i) \neq l^k(x_i)$, то перейти до Кроку 4.
      \item $k=n-1$, якщо $\exists i$ $l^{k+1}(x_i) \neq l^k(x_i)$, то в графі присутній цикл від’ємної довжини і \emph{задача не має розв’язку}. Кінець.
    \end{slim_enumerate}
  \item \emph{Підготовка до наступної ітерації}\\
Оновити мітку наступним чином:
$$S=\{x_i:l^{k+1}(x_i) \neq l^k(x_i)\}$$
  \item Покласти $k=k+1$ і перейти до Кроку 2.
\end{slim_enumerate}

Коли довжини найкоротших шляхів будуть знайдені то самі шляхи отримаємо рекурсивно: $l(x_i')+d(x_i',x_i)=l(x_i), x_i'$ - вершина, що безпосередньо передує $x_i$ на шляху від $x_s$.

\clearpage

\chapter{Знаходження найкоротших шляхів між будь-якими вершинами графа (алгоритм Флойда).}

Цю задачу можна було б розв’язати методом багаторазового використання алгоритму Декстри чи Форда з послідовним перебором кожної вершини графа в ролі початкової, але це вимагало б великої обчислюваної роботи.

Більш ефективним методом розв’язування задачі 3) є алгоритм Флойда. Він застосовується до графів з довільними дугами, але не допускається наявність циклу від’ємної довжини.

В алгоритмі використовуються дві, оновлювані в його процесі, матриці: матриця ваг - $D$ і матриця попередніх вершин - $\Theta$. Прицьому, на $k$-й ітерації елементи матриці ваг $D_{(k)}=\{d_{ij}^{(k)}\}$ позначають найкоротший шлях між вершинами $x_i$ та $x_j$, який може складатись із внутрішніх (проміжних) вершин з множини перших $k$ вершин графу - $\{x_1, ..., x_k\}$ а елементи матриці $\Theta=\{\theta_{ij}^{(k)}\}$ позначають вершини, що безпосереньо передують вершинам $x_j$ у біжучому найкоротшому шляху від $x_i$ до $x_j$.

\emph{Внутрішня (проміжна) вершина - вершина графу, що не збігається з його початковою або кінцевою вершиною.}

\begin{slim_enumerate}
  \item \emph{Присвоєння початкових значень}\\
Нехай задано зважений граф $G=(V,E)$ i вагова функція $d:E \rightarrow R$.
$k=0$\\
$\forall (i,j) \in E d_{ij}^{(k)}=d(x_i,x_j)$ (є дуги)\\
$\forall (i,j) \notin E d_{ij}^{(k)}=\infty$ (дуги відсутні)\\
$d_{ii}^{(k)}=0$ (діагональні)\\
$\theta_{ij}^{(k)}=x_i$
  \item $k=k+1$
  \item $\forall i: i \neq k, d_{ik}^{(k-1)} \neq \infty, \forall j: j \neq k, d_{kj}^{(k-1)} \neq \infty, d_{ij}^{(k-1)}>d_{ik}^{(k-1)}+d_{kj}^{(k-1)}$ оновити матриці:\\
$d_{ij}^{(k)}=d_{ik}^{(k-1)}+d_{kj}^{(k-1)}$\\
$\theta_{ij}^{(k)}=\theta_{kj}^{(k-1)}$\\
Для всіх інших $i$ та $j$ переписати попередні елементи:\\
$d_{ij}^{(k)}=d_{ij}^{(k-1)}$,\\
$\theta_{ij}^{(k)}=\theta_{ij}^{(k-1)}$.
  \item 
    \begin{slim_enumerate}
      \item $d_{ii}^{(k)} < 0$ - в графі пристуній цикл від’ємної довжини, що містить вершину $x_i$ - розв’язку не існує. Кінець.
      \item $d_{ii}^{(k)} \geq 0, k=n$ - маємо розв’язок - матриця $D^{(n)}$ містить найкоротші шляхи між вершинами графу. Кінець.
      \item $d_{ii}^{(k)} \geq 0, k<n$ - перейти на Крок 2.
    \end{slim_enumerate}
\end{slim_enumerate}

\emph{Зауваження.} Якщо в початковій матриці $D^{(0)}$ усі діагональні елементи покласти рівними $\infty$ то $d_{ii}^{(n)}$ буде рівним вазі ланцюга що проходить через $x_i$.

\clearpage

\chapter{Формулювання транспортної задачі. Властивості транспортної задачі.}

Нехай маємо $m$ пунктів виробництва однорідного продукту (бази, склади, ...) з потужностями, відповідно, $a_i, i = \overline{1, m}$. Маємо $n$ пунктів споживання, відповідно, з потребами $b_j, j =\overline{1, n}$.

Задається матриця перевезень $С = \{c_{ij}\}$, де $c_{ij}$ - вартість перевезення одиниці продукту з $i$-того пункту виробництва в $j$-й пункт споживання.

Потрібно знайти такий набір $x_{ij} \geq 0, i = \overline{1, m}, j = \overline{1, n}$, де  $x_{ij}$ - кількість одиниць продукту, яка перевозиться з $і$-го пункту виробництва в $j$-й пункт споживання, щоб виконувались наступні умови: 
\begin{equation}  \sum_{j=1}^n x_{ij} = a_i, i = \overline{1, m}. \end{equation}
\begin{equation} \sum_{i=1}^m x_{ij} = b_j, j = \overline{1, n}.   \end{equation}
\begin{equation} \sum_{i=1}^m \sum_{j=1}^n c_{ij} x_{ij} \to \min  \end{equation}

Невідємний набір $x_{ij}$, який задовольняє (3.1), (3.2), називається \emph{планом задачі} або \emph{допустимим розвязком}. Той із планів, який надає мінімум в (3.3), називається \emph{оптимальним планом} або розвязком транспортної задачі.

\emph{Зауважимо}, що транспортна задача, поставлена в такій формі, називається "транспортною задачею за критерієм вартості".

Умова \begin{equation}  \sum_{i=1}^n a_i = \sum_{j=1}^m b_j  \end{equation} називається умовою \emph{балансу мас}.

Транспортну задачу зручно зображати таблицею:\\
\begin{tabular}{ | c | c | c | c | c | }
\hline
\diaghead(4,3){easterr}{$c_{1 2}$}{$x_{1 2}$} & \diaghead(4,3){easterr}{$c_{1 2}$}{$x_{1 2}$} & \thead{\vdots} & \diaghead(4,3){easterr}{$c_{1 n}$}{$x_{1 n}$} & \thead{$a_1$} \\
\hline
\diaghead(4,3){easterr}{$c_{2 1}$}{$x_{2 1}$} & \diaghead(4,3){easterr}{$c_{2 2}$}{$x_{2 2}$} & \thead{\vdots} & \diaghead(4,3){easterr}{$c_{2 n}$}{$x_{2 n}$} & \thead{$a_2$} \\
\hline
 \thead{$\cdots$} & \thead{$\cdots$} & \thead{$\ddots$} & \thead{$\cdots$} & \thead{$\cdots$} \\
\hline
\diaghead(4,3){easterr}{$c_{m 1}$}{$x_{m 1}$} & \diaghead(4,3){easterr}{$c_{m 2}$}{$x_{m 2}$} & \thead{\vdots} & \diaghead(4,3){easterr}{$c_{m n}$}{$x_{m n}$} & \thead{$a_m$} \\
\hline
\thead{$b_1$} & \thead{$b_2$} & \thead{\vdots} & \thead{$b_n$} & \thead{} \\
\hline
\end{tabular}

Матричний вигляд умов (3.1), (3.2) - матриця обмежень:\\
\begin{tabular}{ @{\hspace{1.4em}}l l }
$
\setlength{\arraycolsep}{0.27em}
\begin{array}{ccccccccccccc}
A_{1 1} & A_{1 2} & \dots & A_{1 n} & A_{2 1} & A_{2 2} & \dots & A_{2 n} & \dots & A_{m 1} & A_{m 2} & \dots & A_{m n} 
\end{array}$ &  \\
\multicolumn{2}{l}{
$\left(
 \begin{array}{ccccccccccccc}
1 & 1 & \dots & 1 & 0 & 0 & \dots & 0 & \dots & 0 & 0 & \dots & 0 \\
0 & 0 & \dots & 0 & 1 & 1 & \dots & 1 & \dots & 0 & 0 & \dots & 0 \\
\dots & \dots & \dots & \dots & \dots & \dots & \dots & \dots & \dots & \dots & \dots & \dots & \dots \\
0 & 0 & \dots & 0 & 0 & 0 & \dots & 0 & \dots & 1 & 1 & \dots & 1 \\
1 & 0 & \dots & 0 & 1 & 0 & \dots & 0 & \dots & 1 & 0 & \dots & 0 \\
\dots & \dots & \dots & \dots & \dots & \dots & \dots & \dots & \dots & \dots & \dots & \dots & \dots \\
0 & 0 & \dots & 1 & 0 & 0 & \dots & 1 & \dots & 0 & 0 & \dots & 1
\end{array}\right)
\left(\begin{array}{c}
x_{1 1} \\
x_{1 2} \\
\dots \\
x_{1 n} \\
x_{m 1} \\
\dots \\
x_{m n}
\end{array}
\right)
=
\left(\begin{array}{c}
a_1 \\
a_2 \\
\dots \\
a_m \\
b_1 \\
\dots \\
b_n
\end{array}
\right)$}
\end{tabular}

\subsection{Теорема 1}

Ранг матриці обмежень транспортної задачі рівний $r=m+n-1$.

{\bf Доведення:}

Оскільки сума перших рівнянь (3.1) рівна сумі наступних $n$ рівнянь (3.2), тобто виконується умова (3.4), то звідси випливає, що $r \leq m+n-1$.

Для того, щоб показати, що $r=m+n-1$ виділимо в матриці обмежень квадратну матрицю розмірності $m+n-1$ визначник якої нерівний нулю. Для цього можна взяти вектори:\\
\begin{tabular}{ @{\hspace{1em}}l l }
$
\setlength{\arraycolsep}{0.23em}
\begin{array}{cccccccc}
A_{1 n} & A_{2 n} & \dots & A_{m n} &  A_{1 1} & A_{1 2} & \dots & A_{1 n-1} 
\end{array}$ &  \\
\multicolumn{2}{l}{
$\left|
 \begin{array}{cccccccc}
1 & 0 & \dots & 0 & 1 & 1 & \dots & 1 \\
0 & 1 & \dots & 0 & 0 & 0 & \dots & 0 \\
\dots & \dots & \dots & \dots & \dots & \dots & \dots & \dots \\
0 & 0 & \dots & 1 & 0 & 0 & \dots & 0 \\
0 & 0 & \dots & 0 & 1 & 0 & \dots & 0 \\
0 & 0 & \dots & 0 & 0 & 1 & \dots & 0 \\
\dots & \dots & \dots & \dots & \dots & \dots & \dots & \dots \\
0 & 0 & \dots & 0 & 0 & 0 & \dots & 1 
\end{array}\right|$}
\end{tabular}
$\neq 0$

\subsection{Теорема  2}

Для розвязності транспортної задачі необхідно і достатньо, щоб виконувалась умова балансу мас $\sum_{i=1}^m a_i = \sum_{j=1}^n b_j$ .

{\bf Доведення:}

{\it Необхідність.} Нехай $ x_{ij}^*, i = \overline{1, m}, j = \overline{1, n}$ , - розвязок транспортної задачі.

Оскільки  $\sum_{j=1}^n x_{ij}^* = a_i,  i = \overline{1, m};  \sum_{i=1}^m  x_{ij}^* = b_j ,  j = \overline{1, n}$ , то отримаємо $\sum_{i=1}^n a_i = \sum_{j=1}^m b_j$.

{\it Достатність.} Нехай виконується умова балансу мас, покладемо  $x_{ij} = \frac{a_ib_j}{\sum_{i=1}^m a_i}$. Сумуючи це співвідношення по $j$, отримаємо:

$\sum_{j=1}^n x_{ij} = \sum_{j=1}^n \frac{a_ib_j}{\sum_{i=1}^m a_i} = a_i,  i = \overline{1, m}.$

$\sum_{i=1}^m x_{ij} = \sum_{i=1}^m \frac{a_ib_j}{\sum_{i=1}^m a_i} = b_j,  j = \overline{1, n}.$

\clearpage

\chapter{Відкрита та закрита моделі транспортної задачі.}

Модель ТЗ прийнято називати \emph{закритою}, якщо сума обсягів виробництва рівна сумі обсягів споживання, тобто виконується умова балансу мас. Якщо ж виконується якась із нерівностей, то модель ТЗ називається \emph{відкритою}.

Припустимо, що виконується $\sum_{i=1}^m a_i > \sum_{j=1}^n b_j$. В цьому випадку задача формулюється так:
Маємо $m$ пунктів виробництва однорідного продукту і $n$ пунктів споживання, відповідно із запасом $a_i, i=\overline{1,m}$ і потребами $b_j, j=\overline{1,n}$. Задається матриця вартостей перевезень $C=\{cij\}_{m,n}, c_{ij}$ – вартість перевезення одиниці продукту з $i$ в $j$. $x_{ij}$ – кількість продукту, який перевозиться з $i$ в $j$.
Слід знайти такий невідємний набір $x_{ij}$, щоб виконувались умови:
\[ \sum_{j=1}^n x_{ij} \leq a_i, i=\overline{1,m} \]
\[ \sum_{i=1}^m x_{ij} \leq b_j, j=\overline{1,n} \]
При цьому повинен досягатися $\min$ сумарних перевезень:
\[ L = \sum_{i=1}^m \sum_{j=1}^n c_{ij} x_{ij} \to \min \]

Для розвязування цієї задачі треба відкриту модель перетворити в закриту. В даному випадку це здійснюється так:
Вводимо $(n+1)$-й пункт призначення, який є фіктивним, з потребами $b_{n+1} = \sum_{i=1}^m a_i - \sum_{j=1}^n b_j$. При цьому вартості перевезень приймаються рівними $c_{1 n+1} = c_{2 n+2} = \dots = c_{m n+1}$.

Оскільки при рівних вартостях перевезень немає значення, від якого з пунктів виробництва буде скеровуватися продукція до пунктів споживання, то в результаті розв’язування задачі одержимо, що вартість перевезень реальним споживачам буде $\min$, а фіктивному споживачу буде скеровуватися продукція від найменш вигідних пунктів виробництва.

У випадку, коли $\sum_{i=1}^m a_i < \sum_{j=1}^n b_j$, вводимо фіктивний $(m+1)$-й пункт виробництва з обсягами $a_{m+1} = \sum_{j=1}^n b_j - \sum_{i=1}^m a_i$. Якщо нема значення, потреби якого із пунктів споживання не будуть повністю задоволені, то вартості перевезень від фіктивного пункту виробництва приймаються рівними. Якщо потреби якогось із пунктів споживання потрібно повністю задовільнити, то в клітину, що стоїть на перетині рядка з фіктивним пунктом виробництва і цим споживачем, ставиться найбільша вартість перевезення. У звязку з цим потреби цього пункту споживання будуть задоволені за рахунок реальних пунктів виробництва.

\clearpage

\chapter{Опорні плани транспортної задачі та їх властивості.}

План $\{x_{ij}\}_{m,n}$ транспортної задачі називають \emph{опорним планом}, якщо вектори $A_{ij}$ (з матриці обмежень), що відповідають додатним компонентам плану лінійно незалежні.

Оскільки із Т1 $\Rightarrow$ ранг матриці обмедень транспортної задачі $r=m+n-1$, то додатних компонент опорний план може мати не більше ніж $m+n-1$.

\begin{slim_itemize}
  \item Опорний план, який має рівно $m+n-1$ додатних компонент - невироджений.
  \item Опорний план, який має менше $m+n-1$ додатних компонент - вироджений.
\end{slim_itemize}

\emph{Базисом ОП} називається довільна система із $m+n-1$ лінійно незалежних векторів $A_{ij}$, яка містить усі вектори $A_{ij}$, що відповідають додатним компонентам плану.

Поставимо взаємовідповідність між клітинами транспонованої таблиці і векторами $A_{ij}$. Кожній клітині $(i,j) \leftrightarrow A_{ij}$.

Набір клітин $$(i_1,j_1),(i_1,j_2),(i_2,j_2),\dots,(i_s,j_1)$$або$$(i_1,j_1),(i_2,j_1),(i_2,j_2),\dots,(i_1,j_s)$$
називають \emph{ланцюжком}. Звідси видно, що два сусідні елементи ланцюжка лежать або в одному рядку або в одному стовпчику.

\emph{Зауважимо}, що кількість елементів замкненого ланцюжка завжди парна.

\dots табличка з прикладом ланцюжка \dots

\subsection{Критерій лінійної незалежності системи векторів $A_{ij}$ умов транспортної задачі.}

\subsection{Критерій розкладу довільного вектора $A_{kl}$ через систему лінійно незалежних векторів $A_{ij}$ умов транспортної задачі.}

\subsection{Теорема - Критерій опорності плану ТЗ}

Для того щоб план ТЗ був опорним необхідно і достатньо, щоб із клітин, які відповідають додатним перевезенням не можна було скласти замкнений ланцюг.

{\bf Наслідок}

Нехай $B$ - базис опорного плану ТЗ, тоді для довільної клітини $(k,l)$ можна побудувати єдиний ланцюг із елементів множини пар індексів $S=\{(i,s)\}$, що відповідають векторам $B$, який замикається на клітині $(k,l)$.

\clearpage

\chapter{Критерій лінійної незалежності системи векторів $A_{ij}$ умов транспортної задачі.}

Нехай $P$ - довільна система векторів $A_{ij}$ умов транспортної задачі, $I$ - множина пар індексів $(i,j)$, які відповідають векторам $A_{ij} \in P$.

{\bf Теорема:}

Для того, щоб система векторів $P$ була лінійно незалежною необхідно і достатньо, щоб із елементів множини $I$ неможна було скласти замкнений ланцюжок.

{\bf Доведення:}

{\it Необхідність.} $P$ - лінійно незалежна система, покажем, що неможна замкнути ланцюжок. {\it Від супротивного.} Припустимо, що із елементів множини $I$ можна скласти замкнений ланцюжок: $$(i_1,j_1), (i_1,j_2), (i_2,j_2), \dots, (i_s,j_1).$$ Звідси випливає, враховуючи вигляд векторів $A_i$, $$A_{{i_1},{j_1}}-A_{{i_1},{j_2}}+A_{{i_2},{j_2}}-\dots-A_{{i_s},{j_1}}=0,$$ а тому система векторів лінійно залежна, що суперечить вхідній умові.

{\it Достатність.} Припустимо, що замкнений ланцюжок не скласти. Покажемо, що система векторів лінійно незалежна. {\it Від супротивного.} Нехай вектори лінійно залежні. Звідси випливає, що $\exists \alpha_{ij} \neq 0, (i,j) \in I:$ $$\sum_{(i,j) \in I}\alpha_{ij}A_{ij} = 0.$$

Нехай $\alpha_{{i_1}{j_1}} \neq 0$, тоді: 
$$\sum_{(i,j) \in I}\alpha_{ij}A_{ij} = -\alpha_{{i_1}{j_1}}A_{{i_1}{j_1}};$$
$$I_1 = I\setminus\{(i_1,j_1)\}.$$

Компонента $i_1$ вектора в правій частині не рівна нулю, тому в лівій частині існує принаймі один вектор $A_{{i_1}{j_2}}: \alpha_{{i_1}{j_2}}\neq0$, тоді 
$$\sum_{(i,j){\in}I}\alpha_{ij}A_{ij} = -\alpha_{{i_1}{j_1}}A_{{i_1}{j_1}}-\alpha_{{i_1}{j_2}}A_{{i_1}{j_2}}.$$

Оскільки $j_1 \neq j_2$ і $m + j_2$ компонента парвої частини не рівна нулю, то знайдеться принаймі один вектор $A_{{i_2}{j_2}}: \alpha_{{i_2}{j_2}} \neq 0$, тоді $$\sum_{(i,j) \in I}\alpha_{ij}A_{ij} = -\alpha_{{i_1}{j_1}}A_{{i_1}{j_1}}-\alpha_{{i_1}{j_2}}A_{{i_1}{j_2}}-\alpha_{{i_2}{j_2}}A_{{i_2}{j_2}}.$$
і т.д.

Цей процес скінченний, оскільки всі вектори в лівій частині різні, то врезультаті приходимо до:
$$0 = -\alpha_{{i_1}{j_1}}A_{{i_1}{j_1}}-\alpha_{{i_2}{j_1}}A_{{i_2}{j_1}}-\dots-\alpha_{{i_k}{j_k-1}}A_{{i_k}{j_k-1}}, i_k=i_s, 1 \leq s \leq k-2;$$
$$0 = -\alpha_{{i_1}{j_1}}A_{{i_1}{j_1}}-\alpha_{{i_1}{j_2}}A_{{i_1}{j_2}}-\dots-\alpha_{{i_k}{j_{k+1}}}A_{{i_k}{j_{k+1}}}, j_{k+1}=j_l, l \leq 1 \leq k-1.$$

Тоді із елементів $(i_1,j_1), (i_1,j_2), \dots, (i_k,j_k)$ можна скласти замкнений ланцюг:
$$(i_s,j_s), (i_{s+1},j_s), \dots, (i_k = i_s, j_{k-1})$$
$$(i_l,j_l), (i_l,j_{l+1}), \dots, (i_k, j_{k+1} = j_l),$$
що суперечить нашому припущенню.

\clearpage

\chapter{Критерій розкладу довільного вектора $A_{kl}$ через систему лінійно незалежних векторів $A_{ij}$ умов транспортної задачі.}

Нехай система векторів $P$ - лінійно-незалежна, $A_{kl} \notin P$. $I$ - множина пар індексів $(i,j)$, які відповідають векторам $A_{ij} \in P$.

{\bf Теорема}

Вектор $A_{kl}$ можна виразити через вектори системи $P$ тоді і тільки тоді, коли із пар індексів $(i,j)$ можна скласти ланцюг, що замикається на клітині $(k,l)$.

{\bf Доведення:}

{\it Необхідність.} Припустимо, що вектор $A_{kl}$ можна виразити через вектори системи $P$, покажемо, що можливо утворити ланцюг із елементів множини $I$, що замикається на клітині $(k,l)$. з припущення випливає, що система $P'=P \cup \{A_{kl}\}$ - лінійно залежна. Тоді з критерію лінійної незалежності системи векторів-умов ТЗ випливає, що із пар індексів множини $I'=I \cup \{(k,l)\}$ можна побудувати ланцюжок. Причому цей ланцюг проходить через клітину $(k,l)$, так як у протилежному випадку - система векторів $P$ була б лінійно залежною, що суперечить умові.

{\it Достатність.} Візьмемо ланцюг $I'$, що замикається на клітині $(k,l)$ і покажемо, що вектор $A_{kl}$ можна виразити через $P$:
$$(k,j_1),(i_1,j_1),\dots,(i_s,l),(k,l).$$
Враховуючи вигляд векторів $A_{ij}$ можна записати:
$$A_{kj_1} - A_{i_1j_1} + \dots + A_{i_sl} - A_{kl} = 0.$$
Звідси видно, що $A_{kl}$ - лінійна комбінація векторів $A_{ij} \in P$.

\clearpage

\chapter{Побудова початкових опорних планів транспортної задачі. Метод мінімального елемента.}

При побудові початкового ОП методом Пн.-Зх. кута ми не враховували елементів матриці вартостей. Природно сподіватися, що коли враховувати елементи матриці вартостей, то отримаємо ОП кращий від попереднього, тобто затрати на перевезення будуть меншими. Таким методом є метод мінімального елементу.

Серед елементів матриці вартостей шукаємо мінімальний. Припустимо це елемент $c_{kl}$. Заповнення таблиці починаємо із клітини $(k,l)$, аналогічно як в попередньому методі.
$$x_{kl} = \min\{a_k,b_l\},$$
при цьому, або рядок, або стовпчик із наступного розгляду викреслюємо. Відповідно змінюємо або запас в $k-$му пункті постачання, або потреби $l-$го пункту споживання.

В матриці, що залишається знову шукаєм мінімальний елемент і т.д.

\begin{tabular}{ | c | c | c | c | c |}
\hline
\diagcell{4}{3}{8}{3}	&	\diagcell{4}{3}{}{5}	&	\diagcell{4}{3}{1}{4}	&	\diagcell{4}{3}{}{7}		&	9\\
\hline
\diagcell{4}{3}{}{7}	&	\diagcell{4}{3}{}{8}	&	\diagcell{4}{3}{5}{9}	&	\diagcell{4}{3}{10}{11}	&	15\\
\hline
\diagcell{4}{3}{}{4}	&	\diagcell{4}{3}{10}{6}	&	\diagcell{4}{3}{13}{8}	&	\diagcell{4}{3}{}{14}		&	23\\
\hline
8	&	10	&	19	&	10		&\\
\hline
\end{tabular}

\clearpage

\chapter{Побудова початкових опорних планів транспортної задачі. Метод Фогеля.}

В кожному рядку шукаємо мінімальний елемент і наступний за ним по величині. Різницю записуємо справа від рядка. Аналогічно поступаєво із стовпчиками, записуємо внизу кожного.

Серед отриманих різниць шукаємо максимальну і в стовпчику чи рядку, якому віповідає максимальна різниця шукаємо мінімальний елемент матриці вартостей.

Заповнення транспортної таблиці починаємо з отриманої клітини аналогічно як і в попередніх методах.

\begin{tabular}{ | c | c | c | c | c |}
\hline
\diagcell{4}{3}{}{3}	&	\diagcell{4}{3}{}{5}	&	\diagcell{4}{3}{9}{4}	&	\diagcell{4}{3}{}{7}		&	9\\
\hline
\diagcell{4}{3}{}{7}	&	\diagcell{4}{3}{}{8}	&	\diagcell{4}{3}{5}{9}	&	\diagcell{4}{3}{10}{11}	&	15\\
\hline
\diagcell{4}{3}{8}{4}	&	\diagcell{4}{3}{10}{6}	&	\diagcell{4}{3}{5}{8}	&	\diagcell{4}{3}{}{14}		&	23\\
\hline
8	&	10	&	19	&	10		&\\
\hline
\end{tabular}

\clearpage

\chapter{Двоїста задача. Умови оптимальності.}

ТЗ є задачею лінійного програмування, а тому можемо записати двоїсту до неї.

Запишемо вихідну ТЗ.

\begin{equation} \sum_{i=1}^m \sum_{j=1}^n c_{ij} x_{ij} \to \min \label{eq:tplf} \end{equation}
\begin{equation}  \sum_{j=1}^n x_{ij} = a_i, i = \overline{1, m}. \label{eq:tpprod} \end{equation}
\begin{equation} \sum_{i=1}^m x_{ij} = b_j, j = \overline{1, n}. \label{eq:tpcust} \end{equation}
\begin{equation} x_{ij} \geq 0, i = \overline{1, m}, j = \overline{1, n} \label{eq:tpx} \end{equation}

Кожному обмеженню (\ref{eq:tpprod}) поставимо у відповідність змінну $u_i, i = \overline{1, m}$.

Кожному обмеженню (\ref{eq:tpcust}) поставимо у відповідність змінну $v_j, j = \overline{1, n}$.

\begin{equation} L^* = \sum_{i=1}^m a_i u_i + \sum_{j=1}^n b_j v_j \to \max \end{equation}
\begin{equation} u_i+v_j \leq c_{ij}, i = \overline{1, m}, j = \overline{1, n} \end{equation}

Матриця обмежень:\\
\begin{tabular}{ @{\hspace{1.4em}}l l }
  \multicolumn{2}{l}{
    $\left(\begin{array}{c}
        $Транспонована матриця обмежень $[m*n \times m+n]\\
      \end{array}\right)
    \left(\begin{array}{c}
        u_1 \\
        \dots \\
        u_m \\
        v_1 \\
        \dots \\
        v_n
      \end{array}\right)
\leq
    \left(\begin{array}{c}
        c_{11} \\
        c_{12} \\
        \dots \\
        c_{mn}
      \end{array}\right)$}
\end{tabular}

Зауважимо, що змінні $u_i$ відповідають обмеженням (\ref{eq:tpprod}), а обмеження (\ref{eq:tpprod}) - відповідають рядкам таблиці ТЗ, тому змінні $u_i$ відповідають рядкам таблиці ТЗ. Аналогічно, змінні $v_j$ відповідають обмеженням (\ref{eq:tpcust}), а (\ref{eq:tpcust}) - відповідають стовпцям таблиці ТЗ, тому $v_j$ відповідають стовпцям таблиці ТЗ.

\subsection{Теорема - Умови оптимальності ТЗ}

План $x_{ij}$ ТЗ - оптимальний, тоді і тільки тоді, коли для будь-якого рядка $i$ та кожного стовпчика $j$ знайдуться такі числа-потенціали $u_1,\dots,u_m,v_1,\dots,v_n$, що задовільняють умови:
\begin{equation} u_i+v_j = c_{ij}, x_{ij}>0 \label{eq:dtpopt1}\end{equation}
\begin{equation} u_i+v_j \leq c_{ij}, x_{ij}=0 \label{eq:dtpopt2}\end{equation}

{\bf Доведення:}

{\it Достатність.} Умови (\ref{eq:dtpopt1}) і (\ref{eq:dtpopt2}) виконуються. Візьмемо довільний план $\tilde{x_{ij}}$, тоді:\\
\[ L(\tilde{x}_{ij}) = \sum_{i=1}^m \sum_{j=1}^n c_{ij} \tilde{x}_{ij} \geq
 \sum_{i=1}^m \sum_{j=1}^n (u_i+v_j) \tilde{x}_{ij} = 
 \sum_{i=1}^m \sum_{j=1}^n \tilde{x}_{ij} u_i + \sum_{i=1}^m \sum_{j=1}^n  \tilde{x}_{ij} v_j = \]
\[ = \sum_{i=1}^m a_i u_i + \sum_{j=1}^n b_j v_j =
 \sum_{i=1}^m \sum_{j=1}^n x_{ij} u_i + \sum_{i=1}^m \sum_{j=1}^n  x_{ij} v_j =
 \sum_{i=1}^m \sum_{j=1}^n (u_i+v_j) x_{ij} = \]
\[ = \sum_{i=1}^m \sum_{j=1}^n c_{ij} x_{ij} = L(x_{ij}). \]

Отже $x_{ij}$ - оптимальний план.

{\it Необхідність.} $x_{ij}$ - оптимальний план. Оскільки задача (\ref{eq:tplf}) - (\ref{eq:tpx}) (ТЗ) має розв’язок, то в силу 1 теореми двоїстості - двоїста задача також має розв’язок: $u^*_i, i = \overline{1,m}, v^*_j, j= \overline{1,n}$. Звідси випливає, що: $u^*_i+v^*_j \leq c_{ij}, i = \overline{1,m}, v^*_j, j= \overline{1,n}$ - виконується умова (\ref{eq:dtpopt2}).

З 2 теоремою двоїстості $\Rightarrow (u^*_i + v^*_j - c_{ij})x_{ij}=0$. Звідси, якщо $x_{ij}>0$, то $u^*_i + v^*_j = c_{ij}$ - виконується умова (\ref{eq:dtpopt1}).

\clearpage

\chapter{Метод потенціалів розв'язування транспортної задачі.}

\emph{(Базується на умові оптимальності плану ТЗ).}

\begin{slim_enumerate}
  \item \emph{Попередній крок.} Знайти початковий опороний план будь-яким із методів, наприклад методом „Пн-Зх кута“. Припустимо, що план - невироджений.

  \item \emph{Перевірка оптимальності.} Для клітин, які ввійшли в початковий опорний план і які будем називати „заповненими“, використовуючи умову $u_i + v_j = c_{ij}, x_{ij} > 0$, шукаємо числа потенціали $u_i, v_j$. Так як всього невідомих є $n+m$, а заповнених клітин - $m+n-1$, то довільній із змінних, наприклад $u_1$, присвоюємо $0$ і знаходимо всі решту.

Всі інші клітини, які називатимемо "вільними" перевіряєм на умову оптимальності - $u_i + v_j \leq c_{ij}, x_{ij} = 0$:
\begin{equation} \sigma_{ij} = c_{ij} - u_i - v_j \geq 0\end{equation}
Якщо всі вільні клітини задовільняють цю умову - то маємо оптимальний план, в іншому випадку - будемо його покращувати.

  \item \emph{Покращення плану.} Вибрати клітину з найбільшим порушенням умови, нехай це клітина $(k,l)$ - $\sigma_{kl} = \min \sigma_{ij}$.

\emph{Зауваження.} Можна вибирати будь-яку іншу, або першу, клітину з порушенням умови потимальності, але вибір клітини з найбільшим порушенням приводить до оптимального плану швидше.

Із заповнених клітин, будуємо ланцюжок, який замикається на клітині $(k,l)$. Позначаємо клітини почерзі знаками "$+$","$-$", причому клітину $(k,l)$ позначаємо знаком "+". Серед клітин із знаком "$-$" вибираєм клітину із найменшим значенням $x_{ij}$ (перевезенням). Віднімаємо це значення від значень в усіх клітинах із знаком "$-$" та додаємо до значень в клітинах із знаком "$+$". Отримали новий опорний план - повертаємся до перевірки його оптимальності.
\end{slim_enumerate}

\clearpage

\chapter{Метод диференціальних рент.}

Зручний тим, що в процесі розв’язування ми не зустрічаємся з випадком виродженості.

Вважається що ТЗ задана таблицею. В кожному стопці шукаємо мінімальну вартість перевезення і беремо її в кільце (обводимо кільце навколо значення).

Після цього здійснюємо розподіл, причому, заповнюємо тільки ті клітини, які відмічені кільцями. (Порівнюємо обсяг виробництва по відповідному рядку і обсяг споживання по відповідному стовпчику. Менший із порівнюваних обсягів приймаємо за величину перевезення. Перевіряємо чи весь обсяг виробництва розподілений.)

Якщо запас пункту виробництва вичерпаний, а потреби пунктів споживання зв’язаних кільцем з цим пунктом виробництва не є повністю задоволені, то пункт виробництва вважається недостатнім, а рядок \emph{від’ємним}.

Якщо ж в пукнті виробництва є нерозподілений залишок, а всі пункти споживання, пов’язані кільцем із цим пунктом виробництва, є повністю задоволені то пункт вважається надлишковим, а рядок \emph{додатним}.

Якщо не розподілений залишок по рядку є 0, то якщо цей рядок кільцем пов’язаний із додатнім рядком, то знак „$+$“, якщо з від’ємним - „$-$“. Якщо одночасно пов’язаний із від’ємним і додатнім то у відповідному пункті виробництва (довільно) збільшуємо обсяг виробництва і здійснюємо після цього розподіл. Якщо обсяг поставок, по цьому рядку збільшиться то ставим „$-$“, якщо ні - „$+$“.

Надлишок чи недостачу у кожному рядку записуємо зліва у рядку з відповідним знаком.

Наступний етап розв’язування полягає у визначення різниць між найменшою вартістю у додатньому рядку і вартістю у кільці. Ці числа записуємо в додатковому рядку над відповідними стовпчиками.

\emph{Зауважимо.} Якщо кільце стоїть у додатньому рядку, то різниця не обчислюється.

Серед отриманих різниць шукаєм мінімальну, яка називається \emph{проміжна рента} і позначається $d$.

Переходимо до наступної таблиці. До вартостей перевезень у від’ємних рядках додаємо величину проміжної ренти, а вартості у додатніх - не змінюємо.

Знову в кожному стовпці шукаємо мінімальну вартість.

\emph{Зауважимо.} В стовпчику, якому відповідала проміжна рента з’являється ще один мінімальний елемент. Ще мінімальний елемент, в деяких випадках, може появитись в стовпчиках, яким відповідає проміжна рента.

Після цього здійснюється розподіл обсягу вироництв. Оскільки тепер більше елементів в кільці ніж рядків, то розподіл здійснюється по іншому.

Починаємо перегляд по рядках або по стовпчиках, при цьому заповнюємо клітину з кільцем лише тоді, коли вона є єдиною відміченою клітиною в своєму рядку, якщо перегляд по рядках і в стовпці - якщо по стовпцях. При повторному перегляді заповнена клітина вже не враховується.

Знову обчилюємо в кожному рядку недостачу чи надлишок, присвоюєм відповідні знаки, обчислюємо величину проміжної ренти $d$ і переходимо до наступної таблиці. Продовжуємо до того часу, поки нерозподілений залишок стане рівний нулю.

\clearpage

\chapter{Транспортна задача за критерієм часу.}

Виникає тоді, коли вартість перевезення не є істостною, а першочергове значення має час, за який здійснюється перевезення. Наприклад, приперевезенні товарів, які швидко псуються.

{\bf Задача}

Нехай маємо $m$ пунктів виробництва однорідного продукту з обсягами $a_i, i = \overline{1,m}$ і $n$ пунктів споживання з потребами $b_j, j= \overline{1,n}$. При цьому, $\sum_{i=1}^m a_i = \sum_{j=1}^n b_j$. Задається матриця $T=[t_{ij}]_{m,n}, t_{ij}$ - час перевезення продукту з пункту постачання $i$ в пункт споживання $j$.

Потрібно знайти такий набір $x_{ij} \geq 0, i = \overline{1,m}, j = \overline{1,n}$, де $x_{ij}$ - кількість продукту, яка перевозиться із $i$ в $j$,щоб виконувались наступні умови:
\begin{slim_enumerate}
  \item $ \sum_{j=1}^n x_{ij} = a_i, i = \overline{1,m} $
  \item $ \sum_{i=1}^m x_{ij} = b_j, j = \overline{1,n} $
  \item Допустимий план $x_{ij}$ повинен бути оптимальний по часу.
\end{slim_enumerate}

З кожним допустимим планом $x=\{x_{ij}\}_{m,n}$ пов’язаний набір $\{t_{ij}\}_x$, прицьому елемент матриці $T$ належить цьому набору $t_{ij} \in \{t_{ij}\}_x$, якщо $x_{ij}>0$.

Тоді $t_x = \max \{t_{ij}\}_x$ - час за який привозиться $x$.

Тоді умова 3) запишеться у вигляді:
\[ t^*=\displaystyle\min_x \max \{t_{ij}\}_x \]

План $x$, для якого виконується така умова є оптимальним по часу.

\subsection{Алгоритм розв’язування ТЗ за критерієм часу}

\begin{slim_enumerate}
  \item {\it Початковий крок.} Яким-небудь із способів будуємо початковий опорний план $x_0$.

  \item {\it Загальний крок.} Визначимо $t'=\displaystyle\max_{x_{ij}>0}\{t_{ij}\}$.

Закреслюємо всі вільні клітини, для яких $t_{ij} \geq t'$.

Будемо покращувати план $x_0$. Стараємося перевезення $x'_{ij}$, що відповідає $t'$, по можливості, зробити нулем. Це дає змогу отримати оптимальний план за меншу кількість кроків.

Для заповнення клітини, що відповідає $t'$ будуємо ланцюжок (замкнутий контур) починаючи із цієї клітини, якій присвоюємо знак „$-$“, при цьому, від’ємними беремо клітини з $x_{ij}>0$, а додатними з $t_{ij}<t'$.

{\it Зауважимо.} Це відбувається так як і в методі потенціалів. Серед від’ємних клітин шукаємо мінімальне перевезення і віднімаєм від від’ємних і додаєм до додатніх.

Переходим до наступної транспортної таблички.

Повторюємо загальний крок до того часу, поки не можливо буде побудувати потрібний ланцюжок. Тоді останнє $t'=t^*$ - мінімальний час перевезення, а останній план - оптимальний.
\end{slim_enumerate}

\clearpage

\chapter{Задача про призначення.}

Нехай маємо $n$ робітників і $n$ робіт, причому кожен робітник може виконувати будь-яку, але одну роботу.

Задається матриця вартостей $C=[c_{ij}]_{n,n}, c_{ij}$ - наприклад, час виконання $i$-м робітником $j$-ї роботи. Позначимо через $x_{ij}$ змінну, яка = 1, якщо $i$-й робітник виконує $j$-у роботу і 0 - в протилежному випадку.

Потрібно так розприділити робітників на виконання робіт, щоб сімарний час виконання всіх робіт був найменший. Тоді математична модель задачі про призначення має наступний вигляд:
\[ \sum_{i=1}^n \sum_{j=1}^n c_{ij} x_{ij} \to \min \]
\[ \sum_{j=1}^n x_{ij} = 1, i = \overline{1,n} \]
\[ \sum_{i=1}^n x_{ij} = 1, j = \overline{1,n} \]
\[ x_{ij} = x_{ij}^2 \mbox{ або (що еквівалентно) } x_{ij} \in \{0,1\} \]
Якщо остатню умову замінити на $x_{ij} \geq 0$ то задачу про призначення можна розглядати як ТЗ при $a_i=1, i = \overline{1,n}, b_j=1, j = \overline{1,n}$.

Оскільки всі $a_i, b_j \in \mathbf{Z}$ то розв’язок також буде цілим.

Задача про призначення є повністю виродженим випадком ТЗ, а тому для її розв’язування розроблений більш ефектнивний метод ніж для розв’язування ТЗ. Цей метод базується на 2-х наступних теоремах.

\section{Теорема 1}

Якщо набір $x_{ij}, i,j=\overline{1,n}$ мінімізує лінійну форму $ L = \sum_{i=1}^n \sum_{j=1}^n c_{ij} x_{ij} $, за умов:
\[ \sum_{j=1}^n x_{ij} = 1, i = \overline{1,n},  \sum_{i=1}^n x_{ij} = 1, j = \overline{1,n}, \]
то він мінімізує і форму:
\[ L' = \sum_{i=1}^n \sum_{j=1}^n (c_{ij} - u_i - v_j) x_{ij}. \]

{\bf Доведення:}

\[ L' = \sum_{i=1}^n \sum_{j=1}^n (c_{ij} - u_i - v_j) x_{ij} = \]
\[ = \sum_{i=1}^n \sum_{j=1}^n c_{ij} x_{ij} - \sum_{i=1}^n \sum_{j=1}^n u_i x_{ij} - \sum_{i=1}^n \sum_{j=1}^n v_j x_{ij} = \]
\[ = L - \sum_{i=1}^n u_i (\sum_{j=1}^n x_{ij}) - \sum_{j=1}^n v_j (\sum_{i=1}^n x_{ij}) = \]
\[ = L - \sum_{i=1}^n u_i - \sum_{j=1}^n v_j \]

Оскільки два останні доданки не залежать від $x$ то мінімум $L$ і $L'$ досягається на тих самих наборах $x_{ij}$.

\section{Теорема 2}

Якщо всі $c_{ij} \geq 0$ і $ \sum_{i=1}^n \sum_{j=1}^n c_{ij} x_{ij} = 0$, то набір $x_{ij}$ є оптимальним планом.

{\bf Очевидно.}

\section{Алгоритм}

\begin{slim_enumerate}
  \item {\it Попередній крок.} Здійснюємо приведення матриці. В кожному рядку шукаєм мінімальний елемент і віднімаєм його від усіх елементів цього рядка. Аналогічно поступаємо із стовпчиками.
  \item {\it Загальний крок.} Мінімальний числом горизонтальних і вертикальних ліній перетинаємо всі нулі принаймі один раз.

Можна показати, що в матриці $n \times n$ можна перетяти всі нулі меншим числом ліній ніж $n$ тоді і лише тоді, коли серед нулів розв’язку нема. Таким чином, якщо всі нулі можна перетяти лише кількістю ліній, що рівна $n$ то маємо розв’язок і його знаходимо. Якщо кількість ліній менша ніж $n$ переходимо до кроку 3.

  \item {\it Перехід.} Серед незакреслених елементів шукаєм мінімальний і віднімаємо його від усіх незакреслених і додаємо до двічі закреслених. Переходимо до кроку 1.
\end{slim_enumerate}

\clearpage

\chapter{Задача про максимальний потік. Теорема Форда-Фалкерсона.}

Нехай на площині маємо $n+2$ точки $P_0, P_1, \dots, P_{n+1}$, при цьому деякі впорядковані пари $P_i, P_j$ з’єднані ланкою $(P_i, P_j)$ так, що утворюють зв’язний ланцюг (сітку).

Ланки $(P_i, P_j), (P_j, P_i)$ - симетричні.

По шляхах $\mu(P_0, P_{i_1}, P_{i_2}, \dots, P_n, P_{n+1})$, що складаються із ланок \[(P_0, P_{i_1}), (P_{i_1}, P_{i_2}), \dots, (P_n, P_{n+1})\] і не утворюють петель, рідина, газ або транспорт із точки $P_0$ - входу сітки, поступає в $P_{n+1}$ - вихід сітки.

Кожній ланці $(P_i, P_j)$ ставиться у відповідність число $a_{ij} \geq 0$, яке називається \emph{пропускною здатністю ланки} і означає кількість речовини, яку може пропустити ця ланка за одиницю часу.

\emph{Потоком} $x_{ij}$ \emph{по ланці} $(P_i, P_j)$ називають кількість речовини, яка проходить по цій ланці за одиницю часу.

Вважаємо (логічно), що потоки задовільнають умовам:
\begin{equation} 0 \ leq x_{ij} \leq a_{ij}, i,j = \overline{0, n+1}; \label{eq:flowvol} \end{equation}
\begin{equation} \sum_{k=0}^n x_{ki} = \sum_{k=1}^{n+1} x_{ik}, i = \overline{1, n}; \label{eq:flowbalance} \end{equation}

Умова (\ref{eq:flowvol}) означає, що величина потоку в ланці не може перевищувати пропускної здатності цієї ланки.

Умова (\ref{eq:flowbalance}) означає, що кількість речовини, яка поступає в будь-яку точку сітки, крім $P_0$-входу і $P_{n+1}$-виходу сітки збігається із кількістю речовини, яка виходить із цієї точки.

З (\ref{eq:flowbalance}) випливає:
\begin{equation} \sum_{i=1}^{n+1} x_{0 i} = \sum_{i=0}^{n} x_{i n+1} = Z \label{eq:flowlineform} \end{equation}

Лінійна форма $Z$ називається \emph{величиною потоку в сітці}.

Ставиться задача знайти величину максимального потоку в сітці, тобто знайти такий набір $x_{ij}^*$, який задовольняє (\ref{eq:flowvol}),(\ref{eq:flowbalance}) і максимізує лінійну форму (\ref{eq:flowlineform}).

{\it Зауважимо.} Задача (\ref{eq:flowvol} - \ref{eq:flowlineform}) є задачею лінійного програмування і, взагалі кажучи, може бути розв’язана симплекс методом.

Перш ніж розглядати більш ефективний алгоритм введемо деякі попередні поняття.

Всі точки сітки розіб’ємо на дві множини $U, V$, які не перетинаються і $P_0 \in U, P_{n+1} \in V$.

Розглянемо множину ланок, що виходять із $U$ і входять у $V$ і назвемо її \emph{перерізом сітки}, позначимо $(U, V)$.

\emph{Пропускною здатністю перерізу} називається величина
\begin{equation} A(U, V) = \sum_{P_i \in U, P_j \in V} a_{ij}. \end{equation}

Очевидно, що для довільного потоку $Z$ і довільного перерізу $(U, V)$ виконується $Z \leq A(U, V)$.

Переріз із найменшою пропускною здатністю називається \emph{мінімальним перерізом}.

\section{Алгоритм}

\begin{slim_enumerate}
  \item {\it Попередній крок.} Записати умови задачі табличкою:\\
\begin{tabular}{ | c | c | c | c | c | c | c | c | }
\hline
	&	$P_0$	&	$\dots$	&	$P_i$	&	$\dots$	&	$P_j$	&	$\dots$	&	$P_{n+1}$\\
\hline
$P_0	$&		&		&	$a_{0 i}$	&		&	$a_{0 j}$	&		&	$a_{0 n+1}$\\
\hline
$\vdots$	&		&		&		&		&		&		&\\
\hline
$P_i$	&	$a_{i 0}$	&		&		&		&		&		& $a_{i n+1}$\\
\hline
$\vdots$	&		&		&		&		&		&		&\\
\hline
$P_j$	&	$a_{j 0}$	&		&		&		&		&		& $a_{j n+1}$\\
\hline
$\vdots$	&		&		&		&		&		&		&\\
\hline
$P_{n+1}$	&	$a_{n+1 0}$	&		&	$a_{n+1 i}$	&		&	$a_{n+1 j}$	&		&\\
\hline
\end{tabular}

В клітину $(P_i, P_j)$ проставити пропускні здатності $a_{ij}>0$. Якщо $a_{ij}=0$ то цей 0 записуєм в таблицю, а якщо $P_l, P_m$ не з’єднані ланкою, то $a_{l m} = 0$, але цей 0 не заповнюється, всі $a_{ii}=0$ теж не заповнюються.

Задача полягає у відшуканні всіх можливих шляхів із $P_0$ в $P_{n+1}$ і знаходженні їх пропускних здатностей.

  \item {\it Загальний $k$-й крок.}
  \begin{slim_enumerate}
    \item {\it Відшукання шляху з одночасним визначенням його пропускної здатності.}

Так як, шукаєм шлях від початкової вершини $P_0$, то біля $P_0$ і в рядку і в стовпці записуєм * - позначення.
    \begin{slim_enumerate}
      \item По відміченому рядку довільно вибираєм комірку, в якій значення відмінне від нуля $(a_{0 i_0} \neq 0)$, і стовпчику цієї комірки присвоюєм мітку $(0, h_{i_0})$, де перше значення - вершина з якої прийшли (рядок в якому шукали, $P_0$), друге - пропускна здатність шляху по якому йдем, так як це перша ланка шляху, то поки пропускна здатність шляху = пропускній здатності ланки $(h_{i_0}=a_{0 i_0})$.
      \item Аналогічне позначення пишем в рядку, що відповідає даному стовпцю. В цьому рядку повторюєм крок {\bf 3.a.} - шукаєм комірку зі значенням відмінним від нуля $(a_{i_0 j} \neq 0)$. Стовпцю вибраної комірки $(a_{i_0 j})$ знову присвоюємо мітку $(i_0, h_j)$, де $i_0$ - знову, вершина з якої прийшли (рядок в якому шукали), а $h_j$ - пропускна здатність шляху $= \min \{h_{i_0}, a_{i_0 j}\}$ - тобто менше між пропускною здатністю попередньої частини шляху і пропускною здатністю даної ланки.
      \item Продовжуємо поки не знайдем шлях між $P_0$ і $P_{n+1}$, тобто не надпишем мітку над стовпчиком з останньою вершиною. Далі виписуєм цей шлях $(\mu_k(P_0, \dots , P_{n+1}), k \mbox{ - лічильник шляхів})$ і його пропускну здатність - $\theta_k=h_{n+1}$ - значення з останньої мітки.

Якщо процес відмічання неможливо продовжити - тобто не існує більше шляхів із $P_0$ в $P_{n+1}$ - алгоритм закінчує свою роботу.
    \end{slim_enumerate}
  \item {\it Визначення нових пропускних здатностей ланок знайденого шляху та симетричних з ними.}

По мітках стовпців знаходимо комірки, по яких ми „пройшлись“ і проставляєм позначки „$-$“, а в симетричних до них - „$+$“. Від значень в комірках з „$-$“ віднімаєм пропускну здатність знайденого шляху ($\theta_k$), а до значень в комірках з „$+$“ додаєм. Отримали нову табличку вертаємся до кроку {\bf (a.i)}.
  \end{slim_enumerate}
\end{slim_enumerate}

Для визначення величини потоків по ланках від елементів вихідної таблиці віднімаємо відповідні елементи останньої таблиці. Додатні значення різниць і є величинами потоків по шляхах.

\section{Теорема Форда-Фалкерсона.}

Величина максимального потоку в сітці не перевищує пропускної здатності мінімального перерізу, причому існує потік величина якого рівна пропускній здатності мінімального перерізу.

{\bf Доведення.}

Покажемо, що якщо по ланці є потік, то по симетричній ланці немає потоку.

Припустимо, що по ланці $(P_i, P_j)$ є потік $x_{ij}>0$. нехай сумарна кількість речовини, що проходить по ланці $(P_i, P_j)$ становить $\theta'$. Тоді пропускна здатність цієї ланки становить $a_{ij} - \theta'$, а симетричної до неї: $a_{ji} + \theta'$.

Припустимо, що кількість речовини, що проходить через $(P+j, P_i)$ становить $\theta''$, тоді пропускна здатність $(P_j, P_i)$ становить $a_{ji}+\theta'-\theta''$, а $(P_i, P_j)$: $a_{ij}-\theta'+\theta''$.

Тим самим, отримуємо:
\[ x_{ij} = a_{ij} - (a_{ij} - \theta' + \theta'') = \theta' - \theta'' > 0 \]
\[ x_{ji} = a_{ji} - (a_{ji} + \theta' - \theta'') = \theta'' - \theta' < 0 \mbox{, що неможливо.}\]

Покажемо, що отримана з алгоритму величина потоку по сітці - оптимальна.

Алгоритм закінчує роботу, якщо ми не можем знайти нового шляху із $P_0$ в $P_{n+1}$, тобто отримуємо таблицю, в якій наможливо продовжити процес мічення. Розіб’ємо точки сітки на дві підмножити $U^*, V^*$, тобто будуємо переріз $(U^*, V^*)$ наступним чином: $P_0 \in U^*$, вершина $P_k \in U^* \Leftrightarrow$ її можня досягти із т. $P_0$, тобто $\exists \mu(P_0, P_{i_1}, \dots, P_k)$. Всі решта точок віднесемо до $V^*$, отже $P_{n+1} \in V^*$.

Розглянемо ланку $(P_i, P_j): P_i \in U^*, P_j \in V^*$. Отримуємо: $x_{ij}=a_{ij}$ і $x_{ji}=0$ або $x_{ij}=0$, оскільки $a_{ij}=0, a_{ji} \neq 0$, але ланка $(P_j, P_i)$ не використовується.

Звідси отримуємо, що: 
\[ Z= \sum_{P_i \in U^*, P_j \in V^*} a_{ij}, \]
з другого боку
\[ A(U^*, V^*) = \sum_{P_i \in U^*, P_j \in V^*} a_{ij}. \]
Оскільки для довільного потоку і перерізу справедливо $Z \leq A(U, V)$, то отримуємо, що 
\[ Z = A(U^*, V^*), \]
\[ Z^*=Z, x_{ij}^*=x_{ij}, \]
\[ Z^*=\min_{(U,V)} A(U,V). \mbox{ Доведено.}\]

\clearpage

\chapter{Метод гілок і меж. Загальна схема.}

\clearpage

\chapter{Задача комівояжера.}

Комівояжер повинен побувати в ряді міст. Задаються віддалі, час або вартість проїзду між кожною парою міст. Комівояжер повинен, відповідно, вибрати найкоротний, найменш тривалий, найдешевший, замкнутий маршрут, який 1 раз проходить через кожне місто і завершується в початковому. Якщо віддалі, час або вартість не залежать від напрямку руху, то така задача є \emph{симетрична}.

Нехай $S(0)$ - множина всіх допустимих замкнутих маршрутів (циклів) задачі комівояжера $n \times n$ з матрицею витрат $[c_{ij}]$. $S(0)$ складається з $(n-1)!$ циклів.

Здійснимо зведення матриці витрат, аналогічно як і в задачі про призначення. В наслідок цього в кожному ряку і кожному стовпчику матриці буде 1 нульовий елемент. Якщо серед нулів можна знайти замкнутий маршрут, то він і буде оптимальним. а його вартість рівна сумі звідних констант. Звідси отримуєм, що вартість довільного циклу із множини $S(0)$ не є маншою ніж сума звідних констант. Позначимо суму звідних констант через $r$ і це число буде нижньою оцінкою множини $S(0)$.

Метод гілок та меж полягає в наступному: множина $S(0)$ розбивається на 2 підмножини, що не перетинаються і обчислюються нижні оцінки кожної з підмножин.

Підмножина, оцінка якої менша, відбирається для наступного розгалуження і т.д. поки не отримаєм розв’язок задачі.

\section{Алгоритм}

\begin{slim_enumerate}
  \item {\it Зведення матриці.} В кожному рядку матриці знайти найменше значення (звідна константа) і відняти його від усіх елементів рядка. Аналогічно робим зі стовпчиками.

Отримали зведену матрицю $[c'_{ij}]$, в кожному рядку і кожному стовпці якої маємо принаймі один нуль.

Обчислюємо суму звідних констант r.

  \item Для кожного нульового елемента матриці $[c'_{ij}]$ обчислити штраф за невикористання.

Нехай $(h,k)$ - нульовий елемент. Якщо ребро $(h,k)$ не використовується, то в маршрут обов’язково повинне ввійти ребро, яке виходить з $h$ і ребро, яке заходить в $k$. Тому штраф за невикористання $P_{h k}$ складе не меншне ніж сума мінімального елемента рядка h і мінімального елемента стовпчика k.
\[ P_{h k} = \displaystyle\min_{i \neq h}\{c'_{i k}\} + \displaystyle\min_{j \neq k}\{c'_{h j}\} \]
Штраф записуєм злів зверху кожного нуля.

  \item Вибираєм клітину з найбільшим штрафом (якщо декілька - вибираєм довільно). Припустимо це клітина $(h, k)$. 

Розбиваємо вибрану множину (на {\bf першій ітерації} це множина всіх штрафів - $S(0)$) на дві підмножини, що не перетинаються: $S(h,k)$ - містить $(h,k)$ і $S \overline{(h,k)}$ - не містить ребра $(h, k)$.

  \item Обчислюємо нижні оцінки кожної із підмножин. Нехай $\theta'$ - нижня оцінка множини, яка розбивається (на {\bf першій ітерації} це - $r$).
  \begin{slim_enumerate}
    \item Обчислимо нижню оцінку множини $S \overline{(h,k)}$.

Як вже вказувалось, якщо маршрут не містить ребра $(h,k)$ то додаткові затрати складуть на менше ніж $P_{h k}$, крім суми звідних констант $r$. Отже:
\[ \theta \overline{(h,k)} = \theta'+P_{h k}. \]
    \item Обчислюємо нижню оцінку $S(h,k)$.
    \begin{slim_enumerate}
      \item Якщо $(h,k)$ входить то жодне інше ребро, що починається в $h$ і жодне інше ребро, що входить в $k$ не може входити в маршрут. Тому рядок $h$ і стовпчик $k$ викреслюються.
      \item {\bf На першій ітерації.} Якщо входить ребро $(h,k)$ то ребро $(k,h)$ не може входити, тому $c'_{k h} = \infty$.

{\bf На всіх подальших ітераціях.} Шукаємо елемент $(\alpha, \beta)$, який із $(h,k)$ і раніше включеними в цикл ребрами утворює замкнутий підцикл і, відповідно, вартість $c'_{\alpha \beta}$ приймаємо рівною $\infty$.

      \item Зводимо матрицю, якщо потрібно і обчислюєм суму звідних констант $r_{h k}$.
      \item $ \theta(h,k) = \theta' + r_{h k} $
    \end{slim_enumerate}
  \end{slim_enumerate}
  \item Для наступного розгалуження вибираємо підмножину з меншою оцінкою.
  \begin{slim_enumerate}
    \item Якщо $\theta\overline{(h,k)} < \theta(h,k)$ то вибираємо множину $S\overline{(h,k)}$ і повертаємося до кроку {\bf 2} з матрицею (отриманою до кроку {\bf 4}) $[c'_{ij}], c'_{h k}=\infty$ і отриману матрицю, якщо потрібно зводимо.
    \item Якщо $\theta\overline{(h,k)} > \theta(h,k)$ то обираємо $S(h,k)$ і повертаємося до кроку {\bf 2} з матрицею отриманою по кроці {\bf 4}).
  \end{slim_enumerate}
\end{slim_enumerate}

\clearpage

\chapter{Метод гілок і меж для задачі комівояжера.}

\clearpage

\chapter{Метод гілок і меж для задач лінійного цілочислового програмування.}

\clearpage


\chapter{Планування на мережах. Основні поняття та визначення.}

\clearpage

\chapter{Планування на мережах. Структура та правила побудови.}

\clearpage

\chapter{Основні поняття теорії ігор. Класифікація ігор.}

Гра є математичною моделлю реальної конфліктної ситуації. \emph{Теорія Ігор} - розділ математики, в якому досліджуються питання поведінки і виробляються оптимальні правила, стратегії поведінки для кожного із учасників гри. Гра хаактеризується системою правил, які визначають кількість учасників гри, їх можливі дії та розподіл виграшів в залежності від ситуації, що склалася в процесі проведення гри.

Під гравцем розуміють одного учасника або групу учасників, які мають спільні інтереси. \emph{Стратегією графця} називається набір правил, які вказують, який вибір варіанту дій він повинен зробити в залежності від ситуації, яка склалася в процесі проведення гри.

Стратегія є оптимальною, якщо в разі багато разового проведення гри, вона забезпечить гравцю максимально можливий середній виграш, або мінімально можливий середній програш.

\emph{Ходи} бувають особисі і випадкові.

\subsection*{Класифікація ігор}

В залежності від виду гри розробляються методи її розв'язування. Різні конфліктні ситуації призводять до різних видів ігор.

Основні напрямки класифікації:
\begin{slim_enumerate}
  \item Кількість гравців.
  \begin{slim_itemize}
    \item одно
    \item два
    \item багато
  \end{slim_itemize}
  \item Кількість стратегій.
  \begin{slim_itemize}
    \item скінчені
    \item нескінченні
  \end{slim_itemize}
  Якщо в грі кожен із гравців має скінченну кількість стратегій, то така гра називається скінченною. Якщо принаймні один із гравців маж нескінченну кількість стратегій, то така гра - нескінченна.
  \item Характер взаємозв'язків.
  \begin{slim_itemize}
    \item безкоаліційні
    \item кооперативні
  \end{slim_itemize}
  В безкоаліційній грі гравцям не дозволяється вступати в коаліцію, утворювати угоди і т.д.\\
  В кооперативній - коаліції визначені заздалегідь.
  \item Характер виграшів.
  \begin{slim_itemize}
    \item з нульовою сумою виграшів
    \item з ненульовою сумою виграшів
  \end{slim_itemize}
  В іграх з ненульовою сумою виграшів, загальний капітал гравців не змінюється, а перерозподіляється між гравцями в залежності від результату. Гра двох гравців з нульовою сумою виграшів є антагоністичною, оскільки виграш одного із гравців рівний програшу другого.
  \item Вигляд функції виграшів.
  \begin{slim_description}
    \item[матричні] гра двох гравців з нулевою сумою виграшу. В цьому випадку гра задається матрицею виграшів першого гравця. Виграш першого гравця = програшу другого.
    \item[біматричні] гра двох гравців, в якій виграші кожного із гравців задаються окремими матрицями.
    \item[неперервні] функція виграців - неперервна.
    \item[опуклі] функція виграшів - опукла.
    \item[сепарабельні] функція виграшів задається сумою добутків функцій від однієї змінної.
    \item[типу дуелей] характеризуються моментом вибору та імовірністями розподілу виграшу в залежності від часу, що пройшов від початку гри до моменту вибору.
  \end{slim_description}
  \item Кількість ходів.
  \begin{slim_itemize}
    \item однокрокові
    \item багатокрокові
  \end{slim_itemize}
  Матриця двох гравців з нульовою сумою виграшів є однокроковою. Кожен із гравців робить по одному ходу.
  \item Стан інформації
  \begin{slim_itemize}
    \item з повною інформацією
    \item з неповною інформацією
  \end{slim_itemize}
  Якщо в грі відомо про всі попередні вибари гравців, то така гра є - з повною інформацією.
\end{slim_enumerate}

Класифікація ігор є умовною, можлива й інша.

\clearpage

\chapter{Матрична гра двох гравців із нульовою сумою виграшів. Верхня та нижня ціни гри.}

\subsection*{ШО це ваще таке з реального життя}

\begin{description}
\item[Гра з двома пальцями] \hfill \\
Одночасно незалежно один від одного два гравці показують одина або два пальці і називають цифру 1 або 2 -- кількість пальців, яка на думку гравця показана противником. Якщо обидва вгадали чи не вгадали -- нічия (0-виграш кожному). Якщо один вгадав, а другий - ні, то переможець отримує виграч рівний сумі паліців показаній обидвома гравцями.

При моделюванні цієї ситуації для кожного гравця можливий стан буде задаватись парою: \{a, b\}, де a -- кількість палців, b -- передбачення кількості пальців противника.
Відповідно можем побудувати матрицю

\begin{tabular}{ r | c | c | c | c | }
         & 1, 1 & 1, 2 & 2, 1 & 2, 2 \\ \hline
  1, 1 & 0 & 2 & -3 & 0 \\ \hline
  1, 2 & -2 & 0 & 0 & 3 \\ \hline
  2, 1 & 3 & 0 & 0 & -4 \\ \hline
  2, 2 & 0 & -3 & 4 & 0 \\ \hline
\end{tabular}

\end{description}

\subsection*{Купу теорії}
Кожен із гравців має певну кількість стратегій. Перший - m стратегій, другий - n стратегій.

Задається матриця $A=\{a_{ij}\}_{m, n}$ - де $a_{ij}$ - виграш першого гравця, ща умови, що він вибирає свою $i$-ту стратегію, а другий - $j$-ту.

Кожен із гравців, незалежно один від одного, робить по одному ходу. Перший гравець обирає свою $i$-ту стратегію, а другий - $j$-ту. На цьому гра закінчується.

Якщо $a_{ij}>0$, то другий гравець платить першому виграш у розмірі $a_{ij}$ умовних одиниць.\\
Якщо $a_{ij}<0$, то перший гравець платить другому виграш у розмірі $|a_{ij}|$ умовних одиниць.

Стратегії $i=\overline{1, m}$ та $j=\overline{1, n}$, відповідних гравців називаються чистими стратегіями.

Гра задана, якщо задана матриця виграшів першого гравця.

Кожен з гравців зацікавлений в максимальному виграші. Перший гравець аналізуючи свою матрицю виграшів розуміє, що другий гравець буде вибирати свої найкращі стратегії. У зв'язку з цим, він визначає $\displaystyle \min_j a_{ij} = \alpha_i$ і вибирає таку стратегію
\begin{equation}
	\max_i \alpha_i = \max_i \min_j a_{ij} = \alpha
\end{equation}
Стратегія, яка забезпечить виграш $\alpha$, називається \emph{максимінна} стратегія. $\alpha$ -- чиста нижня ціна гри.

Аналогічно міркує другий гравець, і визначає
\[
	\min_j \max_i a_{ij} = \beta
\]
Стратегія, що забезпечує виграш $\beta$ називається \emph{мінімаксною}. $\beta$ -- чиста верхня ціна гри.

Чисті верхня і нижня ціни гри означають, що перший гравеь при застосуванні своїх чистис стратігій може забезпечити собі виграш $\ge\alpha$, а другий - може не допустити виграш першого гравця $>\beta$

Стратегії $i_0, j_0$, відповідно першого та другого гравця, які забезпечуюють $\alpha = \beta = v$ називається сідловою точкою матриці гри, а $v$ - \emph{чистою ціною гри}.
Стратегія $i_0, j_0$, що відповідає сідловій точці і сідловому елементу $v=a_{i_0j_0}$, називається розв'язком матричної гри.

Очевидно, що $a_{ij_0} \le a_{i_0j_0} \le a_{i_0j},\,i=\overline{1, m},\,j=\overline{1, n}$. Звідси слідує просте правило для відшукання сідлової точки. В кожному рядку матриці шукаємо мінімальний елемент і перевіряємо чи він є максимальнийм в своєму стовпчику.

Якщо сідлова точка існує, то кажуть, що матриця має розв'язок в чистих стратегіях.

\clearpage

\chapter{Матрична гра двох гравців із нульовою сумою виграшів. Максимінні теореми.}

Нехай $f(x, y), x \in A, y \in B$

\begin{description}

\item[Теорема 1] \hfill \\
Якщо існують $\displaystyle \alpha = \max_{x \in A} \min_{y \in B} f(x, y); \beta = \min_{y \in B} \max_{x \in A} f(x, y)$, то $\alpha \le \beta$

\item[Доведення] \hfill \\
Використовуючи означення максимуму та мінімуму можемо записати:
\[
\min_{y \in B}f(x, y) \le f(x, y) \le \max_{x \in A}f(x, y)
\]
\[
\min_{y \in B}f(x, y) \le \max_{x \in A}f(x, y)
\]
Оскільки нерівність справджується при всіх $x$, то при максимальному буде справджуватись теж:
\[
\max_{x \in A} \min_{y \in B}f(x, y) \le \max_{x \in A}f(x, y)
\]
Аналогічно з ігриком в правій частині:
\[
\max_{x \in A} \min_{y \in B}f(x, y) \le \min_{y \in B} \max_{x \in A}f(x, y)
\]
Що і треба було довести.
\end{description}

Точка $(x_0, y_0)$ називається \emph{сідловою}, якщо
\begin{equation}
f(x, y_0) \le f(x_0, y_0) \le f(x_0, y),\, x \in A,\, y \in B.
\end{equation}

Матрицю A в матричній грі можна розглянути як частковий випадок $f(x, y)$, якщо покласти $x=i, y=j, f(x, y) = a_{ij}$.

Тоді з теореми 1 отримуєм, що в матричній грі з матрицею A чиста нижня ціна не перевищує чистої верхньої: $\alpha \le \beta$.

\begin{description}

\item[Теорема 2] \hfill \\
$f(x, y)$ -- дійсна функція двох змінних, $x \in A, y \in B$ та існують
\[
\max_{x \in A} \min_{y \in B}f(x, y); \min_{y \in B} \max_{x \in A}f(x, y)
\]
Для того, щоб $\max\min = \min\max$ необхідно і достатньо, щоб існувала сідлова точка $(x_0, y_0)$. І якщо так точка існує, то
\begin{equation}
f(x_0, y_0)=\max_{x \in A} \min_{y \in B}f(x, y) = \min_{y \in B} \max_{x \in A}f(x, y)
\end{equation}

\item[Доведення] \hfill \\
(Достатність) Нехай $(x_0, y_0)$ -- сідлова точка, отже:
\[
f(x, y_0) \le f(x_0, y_0) \le f(x_0, y),\, x \in A,\, y \in B.
\]
звідси
\begin{equation}
\max_{x \in A} f(x, y_0) \le f(x_0, y_0) \le \min_{y \in B} f(x_0, y)
\end{equation}
\[
\min_{y \in B} \max_{x \in A} f(x, y) \le \max_{x \in A} f(x, y_0) \le f(x_0, y_0) \le \min_{y \in B} f(x_0, y) \le \max_{x \in A}  \min_{y \in B} f(x, y)
\]
Звідси
\[
\min_{y \in B} \max_{x \in A} f(x, y)  \le \max_{x \in A}  \min_{y \in B} f(x, y)
\]
За теоремою один остання нерівність перетворюється у рівність:
\[
\min_{y \in B} \max_{x \in A} f(x, y)  = \max_{x \in A}  \min_{y \in B} f(x, y)
\]

(Необхідність) 
\[\min_{y \in B} \max_{x \in A} f(x, y)  = \max_{x \in A}  \min_{y \in B} f(x, y)\]
\[\min_{y \in B} \left(\max_{x \in A} f(x, y)\right)  = \max_{x \in A}  \left(\min_{y \in B} f(x, y)\right)\]
\[\min_{y \in B} \left(\max_{x \in A} f(x, y)\right)  = \max_{x \in A} f(x, y_0)\]
\[\max_{x \in A} \left(\min_{y \in B} f(x, y)\right)= \min_{y \in B} f(x_0, y)\]

Покажемо, що $(x_0, y_0)$ -- сідлова точка.
\begin{eqnarray}
\min_{y \in B} f(x_0, y)=\max_{x \in A} f(x, y_0)\\
\min_{y \in B} f(x_0, y) \le f(x_0, y_0)
\end{eqnarray}

Із (5), враховуючи (6), отримаєм:
\[\max_{x \in A} f(x, y_0) \le f(x_0, y_0) \]. Тим самим ми довели ліву нерівність в означені сідлової точки (2). Аналогічно доведеться права нерівність. Отже $(x_0, y_0)$ -- сідлова точка. І враховуючи вихідну умову, отримуємо (3).

\end{description}

\clearpage

\chapter{Оптимальні мішані стратегії та їх властивості.}

Якщо гра не має сідлової точки в чистих стратегіях, то ми можемо визначити лише чисті верхню та нижню ціни гри, які вказують гравцю, що він не може сподіватись на виграш $> \beta$, і може бути впевнений у виграші $\ge \alpha$. Оскільки чиста нижня ціна гри є строго менша чистої верхньої ціни гри (задача не має сідлової точки в чистих стратегіях), то очевидним є бажання першого гравця збільшити свій виграш, а другого -- зменшити свій програш. Пошук такої стратегії приводить до використання так званої мішаної стратегії, яка полягає у випадковому використанні декількох чистих стратегій з певними імовірностями.

Змішаною стратегією гравця називається повний надір імовірностей застосувань його чистих стратегій. Таким чином, якщо перший гравець має $m$ чистиз стратегій, то $x=(x_1, x_2, \dots, x_m)$ є змішаною стратегією першого гравця, якщо $\sum_{i=1}^m x_i = 1; \, x_i \ge 0, i=1, 2, \dots, m$. Аналогічно для другого гравця: $y=(y_1, y_2, \dots, y_n)$ є змішаною стратегією, якщо$\sum_{i=1}^n y_i = 1; \, y_i \ge 0, i=1, 2, \dots, n$.

Зауважимо, що чисту стратегію гри можна розглядати, як частковий випадок мішаної, яка задається одиничним вектором.

Середнім виграшом першого гравця з матрицею виграшів A визначається, як математичне сподівання його виграшів:
\[
M(A, x, y)=\sum_{i=1}^m\sum_{j=1}^n a_{ij}x_i y_i
\]
Формуючи свою стратегію перший гравець повинен орінтуватися на гірше $і$ в зв'язку з цим він оцінює
\[\min_y M(A, x, y)\]
і вибирає таку стратегію $х$, для якої
\[\max_x\min_y M(A, x, y)=\alpha\]
Аналогічно міркує другий і визначає
\[\min_y\max_x M(A, x, y)=\beta\]
При цьому, $\alpha$ -- нижня ціна гри, $\beta$ -- верхня.

Пара мішаних стратегій $x^0, y^0$, відповідно перший і другий гравці називають оптимальними мішаними стратегіями, якщо для них виконується:
\[M(A, x^0, y^0)=\max_x\min_y M(A, x, y)=\min_y\max_x M(A, x, y)\]
при цьому $v = M(A, x^0, y^0)$.

Можна дати інше означення оптимальної мішаної стратегії: мішані стратегії $x^0, y^0$, відповідно першого і другого гравців нащиваються оптимальними, якщо вони утворюють сідлову точку функції $M(A, x, y)$, тобто
\[ M(A, x, y^0) \le M(A, x^0, y^0) \le M(A, x^0, y)\]

Із теореми 2 $\to$ еквівалентність двох означень сідлової точки.

Оптимальні мішані стратегії $x^0, y^0$, та ціна гри $v$ -- називаються розв'язком матричної гри.

Теорема нейма стверджує, що довілна скінчення матрична гра маж розв'язок.

{\bf Теорема 1}(без доведення)
Для матричної гри з матрицею виграшів A $\exists \; \alpha, \beta$:
\[\alpha = \max_x\min_y M(A, x, y), \beta=\min_y\max_x M(A, x, y), \alpha = \beta\]

\subsection*{Властивості оптимальних мішаних стратегій}

{\bf Теорема 2.}
Нехай маємо матричну гру з матрицею виграшів першого гравця - A, тоді для того, шоб змішана стратегія першого гравця $x^0$ була оптималоною $\iff$ щоб для всіх мішаних стратегій другого гравця виконувалось:
\[M(A,x^0,y)\le v\]
Анологічно, для другого: для того, шоб змішана стратегія другого гравця $y^0$ була оптималоною $\iff$ щоб для всіх мішаних стратегій першого гравця виконувалось:
\[M(A,x,y^0)\le v\]
 де $v$ -- ціна гри.

{\bf Наслідок.}
Використовуючи останню теорему отримуємо: „для того, щоб стратегія $x^0$ першого гравця була оптимальною $\iff$ $\sum_{i=1}^m a_{ij} x_i^0 \ge 0, \, j=\overline{1, n}$.

Звідси отримуємо, що мішані стратегії, що задовільняють (1) і (2) будуть відповідати оптимальним стратегіям першого та другого гравців. Таким чином для відшукання оптимальної стратегії розв'язуєм задачі:

(Задача 1)
\[\sum_{i=1}^m a_{ij}x_i \ge v, \; j=\overline{1, n}\]
\[\sum_{i=1}^m x_i = 1; \; x_i \ge 0, i =\overline{1, m} \]

(Задача 2)
\[\sum_{j=1}^n a_{ij}y_j \le v, \; j=\overline{1, m}\]
\[\sum_{i=1}^n y_i = 1; \; y_i \ge 0, i =\overline{1, n} \]

{\bf Теорема 3.}
Якщо $x^0, y^0$ -- оптимальні мішані стратегії, відповідно, першого та другого гравців, то:
\[\sum_{i=1}^m a_{ij} x_i^0 > v \Rightarrow y_j^0 = 0;\]
\[\sum_{i=1}^n a_{ij} y_j^0 > v \Rightarrow x_i^0 = 0.\]

\clearpage

\chapter{Спрощення матричних ігор.}

В багатьох випадках розв'язок матриці ігор можна спростити, шляхом викреслювання дублюючих або завідомо невигідниз стратегій. Якщо елементи деякого рядка (стовпця) матриці рівні відповідним елементам іншого рядка (стовпця) матриці, то такі стратигії називають дублюючими.

Якщо елемент $i$-го рядка матриці $\le$ відповідному елементу іншого рядка матриці, то стратегія $i$ називається завідомо невигідною.

Якщо елемент $j$-го стовпця матриці $\ge$ відповідному елементу іншого стовпця матриці, то стратегія $j$ називається завідомо невигідною.

{\bf Приклад:}
\[ A = \left ( \begin{array}{ccccc}
{\bf 4} & 7 & {\bf 2} & 3 & 4 \\
{\bf 3} & 5 & {\bf 6} & 8 & 5 \\
4 & 4 &  2 & 2 & 8 \\
3 & 6 & 1 & 2 & 4 \\
3 & 5 & 6 & 8 & 9
\end{array} \right ) \to \left (  \begin{array}{cc}
4 & 2 \\
3 & 6
\end{array} \right ) \]

Матриця A називається \emph{кососиметричною}, якщо $a_{ij} = -a_{ji}$.

Матрична гра з кососиметричною матрицею називається \emph{симетричною}.

{\bf Теорема 4.}
Нехай маємо симетричну матричну гру. Тоді $v=0$ і, якщо $x$ -- оптимальна мішана стратегія першого гравця, то вона є і оптимальною мішаною стратегією для другого гравця.

{\bf Теорема 5.}
Нехай маємо матричну гру з матрицею A і ціною гри $v_A$. Якщо $x,y$ -- оптимальні мішані стратегії першого та другого гравців, відповідно, то ці ж стратегії будуть оптимальними для матричної гри з матрицею $B = \{b*a_{ij}+c\}, \, b > 0$ з ціною гри $v_B = b*v_A+c$.

\clearpage

\chapter{Гра порядку $2 \times 2$.}
Задається матрицею $A=\left ( \begin{array}{cc}a_{11}&a_{12}\\a_{21}&a_{22}\end{array}\right)$.

{\bf Теорема 6.}
Припустимо що матрична гра не має сідлової точки в чистих стратегіях. Оптимальні мішані стратегії $x=x(x_1,x_2)$ та $y=(y_1,y_2)$ визначаються так:				
\[\begin{array}{l}
a_{11}x_1+a_{21}x_2=v \\
a_{12}x_1+a_{22}x_2=v \\
x_1+x_2=1	 \\
x_1 \ge 0, x_2 \ge0
\end{array} \quad
\begin{array}{l}
a_{11}y_1+a_{12}y_2=v \\
a_{21}y_1+a_{22}y_2=v \\
y_1+y_2=1	 \\
y_1\ge0,  y_2\ge0
\end{array}\]

{\bf Доведення.}
Із наслідку з теореми 2 отримуємо, що відшукання розв'язків матричної гри з матрицею A зводиться до розв'язування::
\[\begin{array}{l}
a_{11}x_1+a_{21}x_2\ge v \\
a_{12}x_1+a_{22}x_2\ge v \\
x_1+x_2=1	 \\
x_1 \ge 0, x_2 \ge0
\end{array} (1) \qquad 
\begin{array}{l}
a_{11}y_1+a_{12}y_2\le v \\
a_{21}y_1+a_{22}y_2\le v \\
y_1+y_2=1	 \\
y_1\ge0,  y_2\ge0
\end{array} (2)\]
Розлянем спочатку задачу (1):
Припустимо, що обидві нерівності є строгими: 
\[\left.\begin{array}{l}
a_{11}x_1+a_{21}x_2> v \\
a_{12}x_1+a_{22}x_2> v 
\end{array} \right \} \mathop{\Longrightarrow}^{\mbox{за т.3}} y_1=0, y_2=0 \]
що неможливо.

Тоді припустимо що:
\[\left.\begin{array}{l}
a_{11}x_1+a_{21}x_2> v \\
a_{12}x_1+a_{22}x_2 \ge v
\end{array} \right \} \Rightarrow y_1=0 \Rightarrow \left \{
\begin{array}{l}
a_{12}y_2\le v \\
a_{22}y_2\le v
\end{array} \right. \Rightarrow x_1=0\]
Що не можливо, оскільки ми припустили, що гра не має розв'язку в чистих стратегіях.

Перебираючи інші можливості, отримуємо доведення теореми.

\clearpage

\chapter{Гра порядку $2 \times n$.}
Задається матрицею $A=\left ( \begin{array}{cccccc}
a_{11}&a_{12}&\dots&a_{1j}&\dots&a_{1n}\\
a_{21}&a_{22}&\dots&a_{2j}&\dots&a_{2n}
\end{array}\right)$.

Припустимо що матрична гра не має сідлової точки в чистих стратегіях. Використовуючи наслідок із теореми 2, шукаємо мішані стратегії $(x_1, x_2)$ та $(y_1,y_2,\dots,y_n)$, відповідно, першого та другого гревців, розв'язуючи неступні задачі:
\begin{equation}
a_{1j}+a_{2j} \ge v, \, j=\overline{1,n}
\end{equation}

\begin{equation}
\begin{array}{l}
a_{11}y_1+a_{12}y_2+\dots+a_{1j}y_j+\dots+a_{1n}y_n \le v\\
a_{21}y_1+a_{22}y_2+\dots+a_{2j}y_j+\dots+a_{2n}y_n \le v
\end{array}
\end{equation}

\begin{equation}
\begin{array}{c}
x_1+x_2=1\\
0<x_1<1, 0<x_2<1
\end{array}
\quad
\begin{array}{c}
\sum_{i=1}^n y_i=1\\
y_j, j=\overline{1,n}
\end{array}
\end{equation}

Оскільки гра не має розв'язку в чистих стратегіях, то обмеження (2) змінюється рівностіми.
\begin{equation}
\begin{array}{l}
a_{11}y_1+a_{12}y_2+\dots+a_{1j}y_j+\dots+a_{1n}y_n = v\\
a_{21}y_1+a_{22}y_2+\dots+a_{2j}y_j+\dots+a_{2n}y_n = v
\end{array}
\end{equation}

Для розв'язання задачі (1, 3, 4) використаємо графічний метод.

Введемо позначення для лівої частини (1):
\[M_j(x_1) = a_{1j}x_1+a_{2j}x_2; x_2=1-x_1\]
\begin{equation}
M_j(x_1)=(a_{1j}-a_{2j})x_1+a_{2j}, j=\overline{1,n}
\end{equation}

Це є середнім виграшом першого гравця за умови, що він щастосує свою мішану стратегію $(x_1,1-x_1)$ а другий -- свою чисту $j$-ту стратегію.

При кожному $j$ $M_j(x_1)$ є прямою у двовимірному просторі.
%графік%

Другий гравець за рахунок своєї чистої стратегії намагається зменшити виграш першого гравця, в зв'язку з цим позначимо:
$\displaystyle \min_jM_j(x_1)=M(x_1)$ -- нижня границя множини обмеження.

Перший гравець за рахунок своєї стратегії намагається збільшити свій виграш.
\[\max_{x_1} M(x_1) = M(x_1^0)\]
\[v=\max_{x_1} M(x_1)=\max_{x_1} \min_jM_j(x_1)=M(x_1^0)\]

Таким чином ми визначили наближене $x_1^0$ і пару чистих стратегій другого гравця, які дають в перетині точку $M^0$. Для отримання точного розв'язку задачі, розв'язують задачу 2х2. 

\clearpage

\chapter{Гра порядку $m \times 2$.}

\clearpage

\chapter{Розв'язування матричних ігор шляхом зведення до задач лінійного програмування.}

\clearpage

\chapter{Поняття Біматричної гри. Умови рівноваги для біматричної гри.}

\clearpage

\chapter{Розв'язування біматричних ігор.}

\end{document}